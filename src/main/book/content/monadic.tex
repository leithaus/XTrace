
\documentclass[12pt,leqno]{book}
\usepackage{amsmath,amssymb,amsfonts} % Typical maths resource packages
\usepackage{graphicx}                 % Packages to allow inclusion of graphics
\usepackage{color}                    % For creating coloured text and background
\usepackage{hyperref}                 % For creating hyperlinks in cross references
\usepackage{makeidx}                  % For indexing
\usepackage{listings}                 % For code listing
\usepackage{mathpartir}               % For grammars, rules, etc
\usepackage{bcprules}                 % For other kinds of rules
\usepackage{diagrams}                 % For commutative diagrams

\lstloadlanguages{Scala,Java,Haskell,XML,bash,HTML,SQL}

\parindent 1cm
\parskip 0.2cm
\topmargin 0.2cm
\oddsidemargin 1cm
\evensidemargin 0.5cm
\textwidth 15cm
\textheight 21cm

\newtheorem{theorem}{Theorem}[section]
\newtheorem{proposition}[theorem]{Proposition}
\newtheorem{corollary}[theorem]{Corollary}
\newtheorem{lemma}[theorem]{Lemma}
\newtheorem{remark}[theorem]{Remark}
\newtheorem{definition}[theorem]{Definition}


\def\R{\mathbb{ R}}
\def\S{\mathbb{ S}}
\def\I{\mathbb{ I}}

\def\Scala{\texttt{Scala}}
\def\ScalaCheck{\texttt{ScalaCheck}}
\def\Haskell{\texttt{Haskell}}
\def\XML{\texttt{XML}}



\makeindex


\title{Pro Scala: Monadic Design Patterns for the Web}

\author{L.G. Meredith  \\
{\small\em \copyright \  Draft date \today }}

 \date{ }
\begin{document}
\lstset{language=Haskell}
\maketitle
 \addcontentsline{toc}{chapter}{Contents}
\pagenumbering{roman}
\tableofcontents
\listoffigures
\listoftables
\chapter*{Preface}\normalsize
  \addcontentsline{toc}{chapter}{Preface}
\pagestyle{plain}
% The book root file {\tt bookex.tex} gives a basic example of how to
% use \LaTeX \ for preparation of a book. Note that all
% \LaTeX \ commands begin with a
% backslash.

% Each
% Chapter, Appendix and the Index is made as a {\tt *.tex} file and is
% called in by the {\tt include} command---thus {\tt ch1.tex} is
% the name here of the file containing Chapter~1. The inclusion of any
% particular file can be suppressed by prefixing the line by a
% percent sign.


%  Do not put an {\tt end{document}} command at the end of chapter files;
% just one such command is needed at the end of the book.

% Note the tag used to make an index entry. You may need to consult Lamport's
% book~\cite{lamport} for details of the procedure to make the index input
% file; \LaTeX \ will create a pre-index by listing all the tagged
% items in the file {\tt bookex.idx} then you edit this into
% a {\tt theindex} environment, as {\tt index.tex}.

The book you hold in your hands, Dear Reader, is not at all what you expected...



\pagestyle{headings}
\pagenumbering{arabic}

%ch.tex


\chapter{Putting it all together}
\begin{center}
{\small\em The application as a whole}
\end{center}

\input{chapters/nine/putting-it-all-together}
\section{Our web application end-to-end}

TBD
\section{Deploying our application}

\subsection{Why we are not deploying on GAE}
\section{From one web application to web framework}

TBD

% \section{Existence problems}
% We begin with some metamathematics.
% All problems about the existence of maps can be cast into one of the
% following two forms, which are in a sense mutually dual.

% \noindent
% {\bf The Extension Problem}\index{extension problem} \    %%% NB index entry tag
% Given an inclusion $ A \stackrel{i}{\hookrightarrow} X $, and a map
% $ A \stackrel{f}{\rightarrow} Y $,
% does there exist a map $f^{\dagger}:X\to Y$ such that
% $f^{\dagger}$ agrees with $f$ on $A$?

% Here the appropriate source category for maps should be clear from the
% context and, moreover, commutativity through a
% candidate $f^{\dagger}$ is precisely
% the restriction requirement; that is,
% $$f^{\dagger}   :  f^{\dagger}\circ i = f^{\dagger}|_A = f\,. $$
% If such an $f^{\dagger}$ exists\footnote{${}^{\dagger}$ suggests striving
% for perfection, crusading}, then it is called an {\bf
% extension}\index{extension!of a map|bi} of $f$ and is said to {\bf
% extend}\index{extend|bi} $f$. In any diagrams, the presence of
% a dotted arrow or an arrow carrying a ? indicates a pious hope, in no way
% begging the question of its existence. Note that we shall usually
% omit $\circ$ from composite maps.

% \noindent
% {\bf The Lifting Problem}\index{lifting problem} \
% Given a pair of maps $E \stackrel{p}{\rightarrow}B$ and $X \stackrel{f}
% {\rightarrow} B $,
% does there exist a map $f^{\circ} : X \to E$, with
% $pf^{\circ} = f  $?


% That {\em all\/} existence problems about maps are essentially of one
% type or
% the other from these two is seen as follows. Evidently, all existence problems
% are representable by triangular diagrams\index{triangular diagrams} and it
% is easily seen that there are only these six possibilities:
% \begin{center}\begin{picture}(300,70)  %augch2 75
% \put(5,60){\vector(1,0){30}}
% \put(55,60){\vector(1,0){30}}
% \put(135,60){\vector(-1,0){30}}
% \put(185,60){\vector(-1,0){30}}
% \put(235,60){\vector(-1,0){30}}
% \put(285,60){\vector(-1,0){30}}
% \put(0,55){\vector(0,-1){30}}
% \put(50,55){\vector(0,-1){30}}
% \put(100,25){\vector(0,1){30}}
% \put(150,25){\vector(0,1){30}}
% \put(200,55){\vector(0,-1){30}}
% \put(250,55){\vector(0,-1){30}}
% \put(28,33){\small ?}
% \put(78,33){\small ?}
% \put(128,33){\small ?}
% \put(178,33){\small ?}
% \put(228,33){\small ?}
% \put(278,33){\small ?}
% \put(10,3){\bf 1}
% \put(60,3){\bf 2}
% \put(110,3){\bf 3}
% \put(160,3){\bf 4}
% \put(210,3){\bf 5}
% \put(260,3){\bf 6}
% \put(35,55){\vector(-1,-1){30}}
% \put(155,25){\vector(1,1){30}}
% \put(135,55){\vector(-1,-1){30}}
% \put(55,25){\vector(1,1){30}}
% \put(235,55){\vector(-1,-1){30}}
% \put(255,25){\vector(1,1){30}}
% \end{picture}\end{center}



% \begin{figure}
% \begin{picture}(300,220)(0,0)
% \put(-20,-20){\resizebox{20 cm}{!}{\includegraphics{3dpdf}}}
% \put(260,-10){\resizebox{15 cm}{!}{\includegraphics{contpdf}}}
% \put(220,80){$\beta$}
% \put(400,-10){$N$}
% \put(260,170){$\beta$}
% \put(90,15){$N$}
% \end{picture}
% \caption{{\em The log-gamma family of densities with central mean
% $<N> \, = \frac{1}{2}$ as a surface and as a contour plot. }}
% \label{pdf}
% \end{figure}

\newpage

%ch.tex


\chapter{Putting it all together}
\begin{center}
{\small\em The application as a whole}
\end{center}

\input{chapters/nine/putting-it-all-together}
\section{Our web application end-to-end}

TBD
\section{Deploying our application}

\subsection{Why we are not deploying on GAE}
\section{From one web application to web framework}

TBD

% \section{Existence problems}
% We begin with some metamathematics.
% All problems about the existence of maps can be cast into one of the
% following two forms, which are in a sense mutually dual.

% \noindent
% {\bf The Extension Problem}\index{extension problem} \    %%% NB index entry tag
% Given an inclusion $ A \stackrel{i}{\hookrightarrow} X $, and a map
% $ A \stackrel{f}{\rightarrow} Y $,
% does there exist a map $f^{\dagger}:X\to Y$ such that
% $f^{\dagger}$ agrees with $f$ on $A$?

% Here the appropriate source category for maps should be clear from the
% context and, moreover, commutativity through a
% candidate $f^{\dagger}$ is precisely
% the restriction requirement; that is,
% $$f^{\dagger}   :  f^{\dagger}\circ i = f^{\dagger}|_A = f\,. $$
% If such an $f^{\dagger}$ exists\footnote{${}^{\dagger}$ suggests striving
% for perfection, crusading}, then it is called an {\bf
% extension}\index{extension!of a map|bi} of $f$ and is said to {\bf
% extend}\index{extend|bi} $f$. In any diagrams, the presence of
% a dotted arrow or an arrow carrying a ? indicates a pious hope, in no way
% begging the question of its existence. Note that we shall usually
% omit $\circ$ from composite maps.

% \noindent
% {\bf The Lifting Problem}\index{lifting problem} \
% Given a pair of maps $E \stackrel{p}{\rightarrow}B$ and $X \stackrel{f}
% {\rightarrow} B $,
% does there exist a map $f^{\circ} : X \to E$, with
% $pf^{\circ} = f  $?


% That {\em all\/} existence problems about maps are essentially of one
% type or
% the other from these two is seen as follows. Evidently, all existence problems
% are representable by triangular diagrams\index{triangular diagrams} and it
% is easily seen that there are only these six possibilities:
% \begin{center}\begin{picture}(300,70)  %augch2 75
% \put(5,60){\vector(1,0){30}}
% \put(55,60){\vector(1,0){30}}
% \put(135,60){\vector(-1,0){30}}
% \put(185,60){\vector(-1,0){30}}
% \put(235,60){\vector(-1,0){30}}
% \put(285,60){\vector(-1,0){30}}
% \put(0,55){\vector(0,-1){30}}
% \put(50,55){\vector(0,-1){30}}
% \put(100,25){\vector(0,1){30}}
% \put(150,25){\vector(0,1){30}}
% \put(200,55){\vector(0,-1){30}}
% \put(250,55){\vector(0,-1){30}}
% \put(28,33){\small ?}
% \put(78,33){\small ?}
% \put(128,33){\small ?}
% \put(178,33){\small ?}
% \put(228,33){\small ?}
% \put(278,33){\small ?}
% \put(10,3){\bf 1}
% \put(60,3){\bf 2}
% \put(110,3){\bf 3}
% \put(160,3){\bf 4}
% \put(210,3){\bf 5}
% \put(260,3){\bf 6}
% \put(35,55){\vector(-1,-1){30}}
% \put(155,25){\vector(1,1){30}}
% \put(135,55){\vector(-1,-1){30}}
% \put(55,25){\vector(1,1){30}}
% \put(235,55){\vector(-1,-1){30}}
% \put(255,25){\vector(1,1){30}}
% \end{picture}\end{center}



% \begin{figure}
% \begin{picture}(300,220)(0,0)
% \put(-20,-20){\resizebox{20 cm}{!}{\includegraphics{3dpdf}}}
% \put(260,-10){\resizebox{15 cm}{!}{\includegraphics{contpdf}}}
% \put(220,80){$\beta$}
% \put(400,-10){$N$}
% \put(260,170){$\beta$}
% \put(90,15){$N$}
% \end{picture}
% \caption{{\em The log-gamma family of densities with central mean
% $<N> \, = \frac{1}{2}$ as a surface and as a contour plot. }}
% \label{pdf}
% \end{figure}

\newpage

%ch.tex


\chapter{Putting it all together}
\begin{center}
{\small\em The application as a whole}
\end{center}

\input{chapters/nine/putting-it-all-together}
\section{Our web application end-to-end}

TBD
\section{Deploying our application}

\subsection{Why we are not deploying on GAE}
\section{From one web application to web framework}

TBD

% \section{Existence problems}
% We begin with some metamathematics.
% All problems about the existence of maps can be cast into one of the
% following two forms, which are in a sense mutually dual.

% \noindent
% {\bf The Extension Problem}\index{extension problem} \    %%% NB index entry tag
% Given an inclusion $ A \stackrel{i}{\hookrightarrow} X $, and a map
% $ A \stackrel{f}{\rightarrow} Y $,
% does there exist a map $f^{\dagger}:X\to Y$ such that
% $f^{\dagger}$ agrees with $f$ on $A$?

% Here the appropriate source category for maps should be clear from the
% context and, moreover, commutativity through a
% candidate $f^{\dagger}$ is precisely
% the restriction requirement; that is,
% $$f^{\dagger}   :  f^{\dagger}\circ i = f^{\dagger}|_A = f\,. $$
% If such an $f^{\dagger}$ exists\footnote{${}^{\dagger}$ suggests striving
% for perfection, crusading}, then it is called an {\bf
% extension}\index{extension!of a map|bi} of $f$ and is said to {\bf
% extend}\index{extend|bi} $f$. In any diagrams, the presence of
% a dotted arrow or an arrow carrying a ? indicates a pious hope, in no way
% begging the question of its existence. Note that we shall usually
% omit $\circ$ from composite maps.

% \noindent
% {\bf The Lifting Problem}\index{lifting problem} \
% Given a pair of maps $E \stackrel{p}{\rightarrow}B$ and $X \stackrel{f}
% {\rightarrow} B $,
% does there exist a map $f^{\circ} : X \to E$, with
% $pf^{\circ} = f  $?


% That {\em all\/} existence problems about maps are essentially of one
% type or
% the other from these two is seen as follows. Evidently, all existence problems
% are representable by triangular diagrams\index{triangular diagrams} and it
% is easily seen that there are only these six possibilities:
% \begin{center}\begin{picture}(300,70)  %augch2 75
% \put(5,60){\vector(1,0){30}}
% \put(55,60){\vector(1,0){30}}
% \put(135,60){\vector(-1,0){30}}
% \put(185,60){\vector(-1,0){30}}
% \put(235,60){\vector(-1,0){30}}
% \put(285,60){\vector(-1,0){30}}
% \put(0,55){\vector(0,-1){30}}
% \put(50,55){\vector(0,-1){30}}
% \put(100,25){\vector(0,1){30}}
% \put(150,25){\vector(0,1){30}}
% \put(200,55){\vector(0,-1){30}}
% \put(250,55){\vector(0,-1){30}}
% \put(28,33){\small ?}
% \put(78,33){\small ?}
% \put(128,33){\small ?}
% \put(178,33){\small ?}
% \put(228,33){\small ?}
% \put(278,33){\small ?}
% \put(10,3){\bf 1}
% \put(60,3){\bf 2}
% \put(110,3){\bf 3}
% \put(160,3){\bf 4}
% \put(210,3){\bf 5}
% \put(260,3){\bf 6}
% \put(35,55){\vector(-1,-1){30}}
% \put(155,25){\vector(1,1){30}}
% \put(135,55){\vector(-1,-1){30}}
% \put(55,25){\vector(1,1){30}}
% \put(235,55){\vector(-1,-1){30}}
% \put(255,25){\vector(1,1){30}}
% \end{picture}\end{center}



% \begin{figure}
% \begin{picture}(300,220)(0,0)
% \put(-20,-20){\resizebox{20 cm}{!}{\includegraphics{3dpdf}}}
% \put(260,-10){\resizebox{15 cm}{!}{\includegraphics{contpdf}}}
% \put(220,80){$\beta$}
% \put(400,-10){$N$}
% \put(260,170){$\beta$}
% \put(90,15){$N$}
% \end{picture}
% \caption{{\em The log-gamma family of densities with central mean
% $<N> \, = \frac{1}{2}$ as a surface and as a contour plot. }}
% \label{pdf}
% \end{figure}

\newpage

%ch.tex


\chapter{Putting it all together}
\begin{center}
{\small\em The application as a whole}
\end{center}

\input{chapters/nine/putting-it-all-together}
\section{Our web application end-to-end}

TBD
\section{Deploying our application}

\subsection{Why we are not deploying on GAE}
\section{From one web application to web framework}

TBD

% \section{Existence problems}
% We begin with some metamathematics.
% All problems about the existence of maps can be cast into one of the
% following two forms, which are in a sense mutually dual.

% \noindent
% {\bf The Extension Problem}\index{extension problem} \    %%% NB index entry tag
% Given an inclusion $ A \stackrel{i}{\hookrightarrow} X $, and a map
% $ A \stackrel{f}{\rightarrow} Y $,
% does there exist a map $f^{\dagger}:X\to Y$ such that
% $f^{\dagger}$ agrees with $f$ on $A$?

% Here the appropriate source category for maps should be clear from the
% context and, moreover, commutativity through a
% candidate $f^{\dagger}$ is precisely
% the restriction requirement; that is,
% $$f^{\dagger}   :  f^{\dagger}\circ i = f^{\dagger}|_A = f\,. $$
% If such an $f^{\dagger}$ exists\footnote{${}^{\dagger}$ suggests striving
% for perfection, crusading}, then it is called an {\bf
% extension}\index{extension!of a map|bi} of $f$ and is said to {\bf
% extend}\index{extend|bi} $f$. In any diagrams, the presence of
% a dotted arrow or an arrow carrying a ? indicates a pious hope, in no way
% begging the question of its existence. Note that we shall usually
% omit $\circ$ from composite maps.

% \noindent
% {\bf The Lifting Problem}\index{lifting problem} \
% Given a pair of maps $E \stackrel{p}{\rightarrow}B$ and $X \stackrel{f}
% {\rightarrow} B $,
% does there exist a map $f^{\circ} : X \to E$, with
% $pf^{\circ} = f  $?


% That {\em all\/} existence problems about maps are essentially of one
% type or
% the other from these two is seen as follows. Evidently, all existence problems
% are representable by triangular diagrams\index{triangular diagrams} and it
% is easily seen that there are only these six possibilities:
% \begin{center}\begin{picture}(300,70)  %augch2 75
% \put(5,60){\vector(1,0){30}}
% \put(55,60){\vector(1,0){30}}
% \put(135,60){\vector(-1,0){30}}
% \put(185,60){\vector(-1,0){30}}
% \put(235,60){\vector(-1,0){30}}
% \put(285,60){\vector(-1,0){30}}
% \put(0,55){\vector(0,-1){30}}
% \put(50,55){\vector(0,-1){30}}
% \put(100,25){\vector(0,1){30}}
% \put(150,25){\vector(0,1){30}}
% \put(200,55){\vector(0,-1){30}}
% \put(250,55){\vector(0,-1){30}}
% \put(28,33){\small ?}
% \put(78,33){\small ?}
% \put(128,33){\small ?}
% \put(178,33){\small ?}
% \put(228,33){\small ?}
% \put(278,33){\small ?}
% \put(10,3){\bf 1}
% \put(60,3){\bf 2}
% \put(110,3){\bf 3}
% \put(160,3){\bf 4}
% \put(210,3){\bf 5}
% \put(260,3){\bf 6}
% \put(35,55){\vector(-1,-1){30}}
% \put(155,25){\vector(1,1){30}}
% \put(135,55){\vector(-1,-1){30}}
% \put(55,25){\vector(1,1){30}}
% \put(235,55){\vector(-1,-1){30}}
% \put(255,25){\vector(1,1){30}}
% \end{picture}\end{center}



% \begin{figure}
% \begin{picture}(300,220)(0,0)
% \put(-20,-20){\resizebox{20 cm}{!}{\includegraphics{3dpdf}}}
% \put(260,-10){\resizebox{15 cm}{!}{\includegraphics{contpdf}}}
% \put(220,80){$\beta$}
% \put(400,-10){$N$}
% \put(260,170){$\beta$}
% \put(90,15){$N$}
% \end{picture}
% \caption{{\em The log-gamma family of densities with central mean
% $<N> \, = \frac{1}{2}$ as a surface and as a contour plot. }}
% \label{pdf}
% \end{figure}

\newpage

%ch.tex


\chapter{Putting it all together}
\begin{center}
{\small\em The application as a whole}
\end{center}

\input{chapters/nine/putting-it-all-together}
\section{Our web application end-to-end}

TBD
\section{Deploying our application}

\subsection{Why we are not deploying on GAE}
\section{From one web application to web framework}

TBD

% \section{Existence problems}
% We begin with some metamathematics.
% All problems about the existence of maps can be cast into one of the
% following two forms, which are in a sense mutually dual.

% \noindent
% {\bf The Extension Problem}\index{extension problem} \    %%% NB index entry tag
% Given an inclusion $ A \stackrel{i}{\hookrightarrow} X $, and a map
% $ A \stackrel{f}{\rightarrow} Y $,
% does there exist a map $f^{\dagger}:X\to Y$ such that
% $f^{\dagger}$ agrees with $f$ on $A$?

% Here the appropriate source category for maps should be clear from the
% context and, moreover, commutativity through a
% candidate $f^{\dagger}$ is precisely
% the restriction requirement; that is,
% $$f^{\dagger}   :  f^{\dagger}\circ i = f^{\dagger}|_A = f\,. $$
% If such an $f^{\dagger}$ exists\footnote{${}^{\dagger}$ suggests striving
% for perfection, crusading}, then it is called an {\bf
% extension}\index{extension!of a map|bi} of $f$ and is said to {\bf
% extend}\index{extend|bi} $f$. In any diagrams, the presence of
% a dotted arrow or an arrow carrying a ? indicates a pious hope, in no way
% begging the question of its existence. Note that we shall usually
% omit $\circ$ from composite maps.

% \noindent
% {\bf The Lifting Problem}\index{lifting problem} \
% Given a pair of maps $E \stackrel{p}{\rightarrow}B$ and $X \stackrel{f}
% {\rightarrow} B $,
% does there exist a map $f^{\circ} : X \to E$, with
% $pf^{\circ} = f  $?


% That {\em all\/} existence problems about maps are essentially of one
% type or
% the other from these two is seen as follows. Evidently, all existence problems
% are representable by triangular diagrams\index{triangular diagrams} and it
% is easily seen that there are only these six possibilities:
% \begin{center}\begin{picture}(300,70)  %augch2 75
% \put(5,60){\vector(1,0){30}}
% \put(55,60){\vector(1,0){30}}
% \put(135,60){\vector(-1,0){30}}
% \put(185,60){\vector(-1,0){30}}
% \put(235,60){\vector(-1,0){30}}
% \put(285,60){\vector(-1,0){30}}
% \put(0,55){\vector(0,-1){30}}
% \put(50,55){\vector(0,-1){30}}
% \put(100,25){\vector(0,1){30}}
% \put(150,25){\vector(0,1){30}}
% \put(200,55){\vector(0,-1){30}}
% \put(250,55){\vector(0,-1){30}}
% \put(28,33){\small ?}
% \put(78,33){\small ?}
% \put(128,33){\small ?}
% \put(178,33){\small ?}
% \put(228,33){\small ?}
% \put(278,33){\small ?}
% \put(10,3){\bf 1}
% \put(60,3){\bf 2}
% \put(110,3){\bf 3}
% \put(160,3){\bf 4}
% \put(210,3){\bf 5}
% \put(260,3){\bf 6}
% \put(35,55){\vector(-1,-1){30}}
% \put(155,25){\vector(1,1){30}}
% \put(135,55){\vector(-1,-1){30}}
% \put(55,25){\vector(1,1){30}}
% \put(235,55){\vector(-1,-1){30}}
% \put(255,25){\vector(1,1){30}}
% \end{picture}\end{center}



% \begin{figure}
% \begin{picture}(300,220)(0,0)
% \put(-20,-20){\resizebox{20 cm}{!}{\includegraphics{3dpdf}}}
% \put(260,-10){\resizebox{15 cm}{!}{\includegraphics{contpdf}}}
% \put(220,80){$\beta$}
% \put(400,-10){$N$}
% \put(260,170){$\beta$}
% \put(90,15){$N$}
% \end{picture}
% \caption{{\em The log-gamma family of densities with central mean
% $<N> \, = \frac{1}{2}$ as a surface and as a contour plot. }}
% \label{pdf}
% \end{figure}

\newpage

%ch.tex


\chapter{Putting it all together}
\begin{center}
{\small\em The application as a whole}
\end{center}

\input{chapters/nine/putting-it-all-together}
\section{Our web application end-to-end}

TBD
\section{Deploying our application}

\subsection{Why we are not deploying on GAE}
\section{From one web application to web framework}

TBD

% \section{Existence problems}
% We begin with some metamathematics.
% All problems about the existence of maps can be cast into one of the
% following two forms, which are in a sense mutually dual.

% \noindent
% {\bf The Extension Problem}\index{extension problem} \    %%% NB index entry tag
% Given an inclusion $ A \stackrel{i}{\hookrightarrow} X $, and a map
% $ A \stackrel{f}{\rightarrow} Y $,
% does there exist a map $f^{\dagger}:X\to Y$ such that
% $f^{\dagger}$ agrees with $f$ on $A$?

% Here the appropriate source category for maps should be clear from the
% context and, moreover, commutativity through a
% candidate $f^{\dagger}$ is precisely
% the restriction requirement; that is,
% $$f^{\dagger}   :  f^{\dagger}\circ i = f^{\dagger}|_A = f\,. $$
% If such an $f^{\dagger}$ exists\footnote{${}^{\dagger}$ suggests striving
% for perfection, crusading}, then it is called an {\bf
% extension}\index{extension!of a map|bi} of $f$ and is said to {\bf
% extend}\index{extend|bi} $f$. In any diagrams, the presence of
% a dotted arrow or an arrow carrying a ? indicates a pious hope, in no way
% begging the question of its existence. Note that we shall usually
% omit $\circ$ from composite maps.

% \noindent
% {\bf The Lifting Problem}\index{lifting problem} \
% Given a pair of maps $E \stackrel{p}{\rightarrow}B$ and $X \stackrel{f}
% {\rightarrow} B $,
% does there exist a map $f^{\circ} : X \to E$, with
% $pf^{\circ} = f  $?


% That {\em all\/} existence problems about maps are essentially of one
% type or
% the other from these two is seen as follows. Evidently, all existence problems
% are representable by triangular diagrams\index{triangular diagrams} and it
% is easily seen that there are only these six possibilities:
% \begin{center}\begin{picture}(300,70)  %augch2 75
% \put(5,60){\vector(1,0){30}}
% \put(55,60){\vector(1,0){30}}
% \put(135,60){\vector(-1,0){30}}
% \put(185,60){\vector(-1,0){30}}
% \put(235,60){\vector(-1,0){30}}
% \put(285,60){\vector(-1,0){30}}
% \put(0,55){\vector(0,-1){30}}
% \put(50,55){\vector(0,-1){30}}
% \put(100,25){\vector(0,1){30}}
% \put(150,25){\vector(0,1){30}}
% \put(200,55){\vector(0,-1){30}}
% \put(250,55){\vector(0,-1){30}}
% \put(28,33){\small ?}
% \put(78,33){\small ?}
% \put(128,33){\small ?}
% \put(178,33){\small ?}
% \put(228,33){\small ?}
% \put(278,33){\small ?}
% \put(10,3){\bf 1}
% \put(60,3){\bf 2}
% \put(110,3){\bf 3}
% \put(160,3){\bf 4}
% \put(210,3){\bf 5}
% \put(260,3){\bf 6}
% \put(35,55){\vector(-1,-1){30}}
% \put(155,25){\vector(1,1){30}}
% \put(135,55){\vector(-1,-1){30}}
% \put(55,25){\vector(1,1){30}}
% \put(235,55){\vector(-1,-1){30}}
% \put(255,25){\vector(1,1){30}}
% \end{picture}\end{center}



% \begin{figure}
% \begin{picture}(300,220)(0,0)
% \put(-20,-20){\resizebox{20 cm}{!}{\includegraphics{3dpdf}}}
% \put(260,-10){\resizebox{15 cm}{!}{\includegraphics{contpdf}}}
% \put(220,80){$\beta$}
% \put(400,-10){$N$}
% \put(260,170){$\beta$}
% \put(90,15){$N$}
% \end{picture}
% \caption{{\em The log-gamma family of densities with central mean
% $<N> \, = \frac{1}{2}$ as a surface and as a contour plot. }}
% \label{pdf}
% \end{figure}

\newpage

%ch.tex


\chapter{Putting it all together}
\begin{center}
{\small\em The application as a whole}
\end{center}

\input{chapters/nine/putting-it-all-together}
\section{Our web application end-to-end}

TBD
\section{Deploying our application}

\subsection{Why we are not deploying on GAE}
\section{From one web application to web framework}

TBD

% \section{Existence problems}
% We begin with some metamathematics.
% All problems about the existence of maps can be cast into one of the
% following two forms, which are in a sense mutually dual.

% \noindent
% {\bf The Extension Problem}\index{extension problem} \    %%% NB index entry tag
% Given an inclusion $ A \stackrel{i}{\hookrightarrow} X $, and a map
% $ A \stackrel{f}{\rightarrow} Y $,
% does there exist a map $f^{\dagger}:X\to Y$ such that
% $f^{\dagger}$ agrees with $f$ on $A$?

% Here the appropriate source category for maps should be clear from the
% context and, moreover, commutativity through a
% candidate $f^{\dagger}$ is precisely
% the restriction requirement; that is,
% $$f^{\dagger}   :  f^{\dagger}\circ i = f^{\dagger}|_A = f\,. $$
% If such an $f^{\dagger}$ exists\footnote{${}^{\dagger}$ suggests striving
% for perfection, crusading}, then it is called an {\bf
% extension}\index{extension!of a map|bi} of $f$ and is said to {\bf
% extend}\index{extend|bi} $f$. In any diagrams, the presence of
% a dotted arrow or an arrow carrying a ? indicates a pious hope, in no way
% begging the question of its existence. Note that we shall usually
% omit $\circ$ from composite maps.

% \noindent
% {\bf The Lifting Problem}\index{lifting problem} \
% Given a pair of maps $E \stackrel{p}{\rightarrow}B$ and $X \stackrel{f}
% {\rightarrow} B $,
% does there exist a map $f^{\circ} : X \to E$, with
% $pf^{\circ} = f  $?


% That {\em all\/} existence problems about maps are essentially of one
% type or
% the other from these two is seen as follows. Evidently, all existence problems
% are representable by triangular diagrams\index{triangular diagrams} and it
% is easily seen that there are only these six possibilities:
% \begin{center}\begin{picture}(300,70)  %augch2 75
% \put(5,60){\vector(1,0){30}}
% \put(55,60){\vector(1,0){30}}
% \put(135,60){\vector(-1,0){30}}
% \put(185,60){\vector(-1,0){30}}
% \put(235,60){\vector(-1,0){30}}
% \put(285,60){\vector(-1,0){30}}
% \put(0,55){\vector(0,-1){30}}
% \put(50,55){\vector(0,-1){30}}
% \put(100,25){\vector(0,1){30}}
% \put(150,25){\vector(0,1){30}}
% \put(200,55){\vector(0,-1){30}}
% \put(250,55){\vector(0,-1){30}}
% \put(28,33){\small ?}
% \put(78,33){\small ?}
% \put(128,33){\small ?}
% \put(178,33){\small ?}
% \put(228,33){\small ?}
% \put(278,33){\small ?}
% \put(10,3){\bf 1}
% \put(60,3){\bf 2}
% \put(110,3){\bf 3}
% \put(160,3){\bf 4}
% \put(210,3){\bf 5}
% \put(260,3){\bf 6}
% \put(35,55){\vector(-1,-1){30}}
% \put(155,25){\vector(1,1){30}}
% \put(135,55){\vector(-1,-1){30}}
% \put(55,25){\vector(1,1){30}}
% \put(235,55){\vector(-1,-1){30}}
% \put(255,25){\vector(1,1){30}}
% \end{picture}\end{center}



% \begin{figure}
% \begin{picture}(300,220)(0,0)
% \put(-20,-20){\resizebox{20 cm}{!}{\includegraphics{3dpdf}}}
% \put(260,-10){\resizebox{15 cm}{!}{\includegraphics{contpdf}}}
% \put(220,80){$\beta$}
% \put(400,-10){$N$}
% \put(260,170){$\beta$}
% \put(90,15){$N$}
% \end{picture}
% \caption{{\em The log-gamma family of densities with central mean
% $<N> \, = \frac{1}{2}$ as a surface and as a contour plot. }}
% \label{pdf}
% \end{figure}

\newpage

%ch.tex


\chapter{Putting it all together}
\begin{center}
{\small\em The application as a whole}
\end{center}

\input{chapters/nine/putting-it-all-together}
\section{Our web application end-to-end}

TBD
\section{Deploying our application}

\subsection{Why we are not deploying on GAE}
\section{From one web application to web framework}

TBD

% \section{Existence problems}
% We begin with some metamathematics.
% All problems about the existence of maps can be cast into one of the
% following two forms, which are in a sense mutually dual.

% \noindent
% {\bf The Extension Problem}\index{extension problem} \    %%% NB index entry tag
% Given an inclusion $ A \stackrel{i}{\hookrightarrow} X $, and a map
% $ A \stackrel{f}{\rightarrow} Y $,
% does there exist a map $f^{\dagger}:X\to Y$ such that
% $f^{\dagger}$ agrees with $f$ on $A$?

% Here the appropriate source category for maps should be clear from the
% context and, moreover, commutativity through a
% candidate $f^{\dagger}$ is precisely
% the restriction requirement; that is,
% $$f^{\dagger}   :  f^{\dagger}\circ i = f^{\dagger}|_A = f\,. $$
% If such an $f^{\dagger}$ exists\footnote{${}^{\dagger}$ suggests striving
% for perfection, crusading}, then it is called an {\bf
% extension}\index{extension!of a map|bi} of $f$ and is said to {\bf
% extend}\index{extend|bi} $f$. In any diagrams, the presence of
% a dotted arrow or an arrow carrying a ? indicates a pious hope, in no way
% begging the question of its existence. Note that we shall usually
% omit $\circ$ from composite maps.

% \noindent
% {\bf The Lifting Problem}\index{lifting problem} \
% Given a pair of maps $E \stackrel{p}{\rightarrow}B$ and $X \stackrel{f}
% {\rightarrow} B $,
% does there exist a map $f^{\circ} : X \to E$, with
% $pf^{\circ} = f  $?


% That {\em all\/} existence problems about maps are essentially of one
% type or
% the other from these two is seen as follows. Evidently, all existence problems
% are representable by triangular diagrams\index{triangular diagrams} and it
% is easily seen that there are only these six possibilities:
% \begin{center}\begin{picture}(300,70)  %augch2 75
% \put(5,60){\vector(1,0){30}}
% \put(55,60){\vector(1,0){30}}
% \put(135,60){\vector(-1,0){30}}
% \put(185,60){\vector(-1,0){30}}
% \put(235,60){\vector(-1,0){30}}
% \put(285,60){\vector(-1,0){30}}
% \put(0,55){\vector(0,-1){30}}
% \put(50,55){\vector(0,-1){30}}
% \put(100,25){\vector(0,1){30}}
% \put(150,25){\vector(0,1){30}}
% \put(200,55){\vector(0,-1){30}}
% \put(250,55){\vector(0,-1){30}}
% \put(28,33){\small ?}
% \put(78,33){\small ?}
% \put(128,33){\small ?}
% \put(178,33){\small ?}
% \put(228,33){\small ?}
% \put(278,33){\small ?}
% \put(10,3){\bf 1}
% \put(60,3){\bf 2}
% \put(110,3){\bf 3}
% \put(160,3){\bf 4}
% \put(210,3){\bf 5}
% \put(260,3){\bf 6}
% \put(35,55){\vector(-1,-1){30}}
% \put(155,25){\vector(1,1){30}}
% \put(135,55){\vector(-1,-1){30}}
% \put(55,25){\vector(1,1){30}}
% \put(235,55){\vector(-1,-1){30}}
% \put(255,25){\vector(1,1){30}}
% \end{picture}\end{center}



% \begin{figure}
% \begin{picture}(300,220)(0,0)
% \put(-20,-20){\resizebox{20 cm}{!}{\includegraphics{3dpdf}}}
% \put(260,-10){\resizebox{15 cm}{!}{\includegraphics{contpdf}}}
% \put(220,80){$\beta$}
% \put(400,-10){$N$}
% \put(260,170){$\beta$}
% \put(90,15){$N$}
% \end{picture}
% \caption{{\em The log-gamma family of densities with central mean
% $<N> \, = \frac{1}{2}$ as a surface and as a contour plot. }}
% \label{pdf}
% \end{figure}

\newpage

%ch.tex


\chapter{Putting it all together}
\begin{center}
{\small\em The application as a whole}
\end{center}

\input{chapters/nine/putting-it-all-together}
\section{Our web application end-to-end}

TBD
\section{Deploying our application}

\subsection{Why we are not deploying on GAE}
\section{From one web application to web framework}

TBD

% \section{Existence problems}
% We begin with some metamathematics.
% All problems about the existence of maps can be cast into one of the
% following two forms, which are in a sense mutually dual.

% \noindent
% {\bf The Extension Problem}\index{extension problem} \    %%% NB index entry tag
% Given an inclusion $ A \stackrel{i}{\hookrightarrow} X $, and a map
% $ A \stackrel{f}{\rightarrow} Y $,
% does there exist a map $f^{\dagger}:X\to Y$ such that
% $f^{\dagger}$ agrees with $f$ on $A$?

% Here the appropriate source category for maps should be clear from the
% context and, moreover, commutativity through a
% candidate $f^{\dagger}$ is precisely
% the restriction requirement; that is,
% $$f^{\dagger}   :  f^{\dagger}\circ i = f^{\dagger}|_A = f\,. $$
% If such an $f^{\dagger}$ exists\footnote{${}^{\dagger}$ suggests striving
% for perfection, crusading}, then it is called an {\bf
% extension}\index{extension!of a map|bi} of $f$ and is said to {\bf
% extend}\index{extend|bi} $f$. In any diagrams, the presence of
% a dotted arrow or an arrow carrying a ? indicates a pious hope, in no way
% begging the question of its existence. Note that we shall usually
% omit $\circ$ from composite maps.

% \noindent
% {\bf The Lifting Problem}\index{lifting problem} \
% Given a pair of maps $E \stackrel{p}{\rightarrow}B$ and $X \stackrel{f}
% {\rightarrow} B $,
% does there exist a map $f^{\circ} : X \to E$, with
% $pf^{\circ} = f  $?


% That {\em all\/} existence problems about maps are essentially of one
% type or
% the other from these two is seen as follows. Evidently, all existence problems
% are representable by triangular diagrams\index{triangular diagrams} and it
% is easily seen that there are only these six possibilities:
% \begin{center}\begin{picture}(300,70)  %augch2 75
% \put(5,60){\vector(1,0){30}}
% \put(55,60){\vector(1,0){30}}
% \put(135,60){\vector(-1,0){30}}
% \put(185,60){\vector(-1,0){30}}
% \put(235,60){\vector(-1,0){30}}
% \put(285,60){\vector(-1,0){30}}
% \put(0,55){\vector(0,-1){30}}
% \put(50,55){\vector(0,-1){30}}
% \put(100,25){\vector(0,1){30}}
% \put(150,25){\vector(0,1){30}}
% \put(200,55){\vector(0,-1){30}}
% \put(250,55){\vector(0,-1){30}}
% \put(28,33){\small ?}
% \put(78,33){\small ?}
% \put(128,33){\small ?}
% \put(178,33){\small ?}
% \put(228,33){\small ?}
% \put(278,33){\small ?}
% \put(10,3){\bf 1}
% \put(60,3){\bf 2}
% \put(110,3){\bf 3}
% \put(160,3){\bf 4}
% \put(210,3){\bf 5}
% \put(260,3){\bf 6}
% \put(35,55){\vector(-1,-1){30}}
% \put(155,25){\vector(1,1){30}}
% \put(135,55){\vector(-1,-1){30}}
% \put(55,25){\vector(1,1){30}}
% \put(235,55){\vector(-1,-1){30}}
% \put(255,25){\vector(1,1){30}}
% \end{picture}\end{center}



% \begin{figure}
% \begin{picture}(300,220)(0,0)
% \put(-20,-20){\resizebox{20 cm}{!}{\includegraphics{3dpdf}}}
% \put(260,-10){\resizebox{15 cm}{!}{\includegraphics{contpdf}}}
% \put(220,80){$\beta$}
% \put(400,-10){$N$}
% \put(260,170){$\beta$}
% \put(90,15){$N$}
% \end{picture}
% \caption{{\em The log-gamma family of densities with central mean
% $<N> \, = \frac{1}{2}$ as a surface and as a contour plot. }}
% \label{pdf}
% \end{figure}

\newpage

%ch.tex


\chapter{Putting it all together}
\begin{center}
{\small\em The application as a whole}
\end{center}

\input{chapters/nine/putting-it-all-together}
\section{Our web application end-to-end}

TBD
\section{Deploying our application}

\subsection{Why we are not deploying on GAE}
\section{From one web application to web framework}

TBD

% \section{Existence problems}
% We begin with some metamathematics.
% All problems about the existence of maps can be cast into one of the
% following two forms, which are in a sense mutually dual.

% \noindent
% {\bf The Extension Problem}\index{extension problem} \    %%% NB index entry tag
% Given an inclusion $ A \stackrel{i}{\hookrightarrow} X $, and a map
% $ A \stackrel{f}{\rightarrow} Y $,
% does there exist a map $f^{\dagger}:X\to Y$ such that
% $f^{\dagger}$ agrees with $f$ on $A$?

% Here the appropriate source category for maps should be clear from the
% context and, moreover, commutativity through a
% candidate $f^{\dagger}$ is precisely
% the restriction requirement; that is,
% $$f^{\dagger}   :  f^{\dagger}\circ i = f^{\dagger}|_A = f\,. $$
% If such an $f^{\dagger}$ exists\footnote{${}^{\dagger}$ suggests striving
% for perfection, crusading}, then it is called an {\bf
% extension}\index{extension!of a map|bi} of $f$ and is said to {\bf
% extend}\index{extend|bi} $f$. In any diagrams, the presence of
% a dotted arrow or an arrow carrying a ? indicates a pious hope, in no way
% begging the question of its existence. Note that we shall usually
% omit $\circ$ from composite maps.

% \noindent
% {\bf The Lifting Problem}\index{lifting problem} \
% Given a pair of maps $E \stackrel{p}{\rightarrow}B$ and $X \stackrel{f}
% {\rightarrow} B $,
% does there exist a map $f^{\circ} : X \to E$, with
% $pf^{\circ} = f  $?


% That {\em all\/} existence problems about maps are essentially of one
% type or
% the other from these two is seen as follows. Evidently, all existence problems
% are representable by triangular diagrams\index{triangular diagrams} and it
% is easily seen that there are only these six possibilities:
% \begin{center}\begin{picture}(300,70)  %augch2 75
% \put(5,60){\vector(1,0){30}}
% \put(55,60){\vector(1,0){30}}
% \put(135,60){\vector(-1,0){30}}
% \put(185,60){\vector(-1,0){30}}
% \put(235,60){\vector(-1,0){30}}
% \put(285,60){\vector(-1,0){30}}
% \put(0,55){\vector(0,-1){30}}
% \put(50,55){\vector(0,-1){30}}
% \put(100,25){\vector(0,1){30}}
% \put(150,25){\vector(0,1){30}}
% \put(200,55){\vector(0,-1){30}}
% \put(250,55){\vector(0,-1){30}}
% \put(28,33){\small ?}
% \put(78,33){\small ?}
% \put(128,33){\small ?}
% \put(178,33){\small ?}
% \put(228,33){\small ?}
% \put(278,33){\small ?}
% \put(10,3){\bf 1}
% \put(60,3){\bf 2}
% \put(110,3){\bf 3}
% \put(160,3){\bf 4}
% \put(210,3){\bf 5}
% \put(260,3){\bf 6}
% \put(35,55){\vector(-1,-1){30}}
% \put(155,25){\vector(1,1){30}}
% \put(135,55){\vector(-1,-1){30}}
% \put(55,25){\vector(1,1){30}}
% \put(235,55){\vector(-1,-1){30}}
% \put(255,25){\vector(1,1){30}}
% \end{picture}\end{center}



% \begin{figure}
% \begin{picture}(300,220)(0,0)
% \put(-20,-20){\resizebox{20 cm}{!}{\includegraphics{3dpdf}}}
% \put(260,-10){\resizebox{15 cm}{!}{\includegraphics{contpdf}}}
% \put(220,80){$\beta$}
% \put(400,-10){$N$}
% \put(260,170){$\beta$}
% \put(90,15){$N$}
% \end{picture}
% \caption{{\em The log-gamma family of densities with central mean
% $<N> \, = \frac{1}{2}$ as a surface and as a contour plot. }}
% \label{pdf}
% \end{figure}

\newpage



\documentclass[12pt,leqno]{book}
\usepackage{amsmath,amssymb,amsfonts} % Typical maths resource packages
\usepackage{graphicx}                 % Packages to allow inclusion of graphics
\usepackage{color}                    % For creating coloured text and background
\usepackage{hyperref}                 % For creating hyperlinks in cross references
\usepackage{makeidx}                  % For indexing
\usepackage{listings}                 % For code listing
\usepackage{mathpartir}               % For grammars, rules, etc
\usepackage{bcprules}                 % For other kinds of rules
\usepackage{diagrams}                 % For commutative diagrams

\lstloadlanguages{Scala,Java,Haskell,XML,bash,HTML,SQL}

\parindent 1cm
\parskip 0.2cm
\topmargin 0.2cm
\oddsidemargin 1cm
\evensidemargin 0.5cm
\textwidth 15cm
\textheight 21cm

\newtheorem{theorem}{Theorem}[section]
\newtheorem{proposition}[theorem]{Proposition}
\newtheorem{corollary}[theorem]{Corollary}
\newtheorem{lemma}[theorem]{Lemma}
\newtheorem{remark}[theorem]{Remark}
\newtheorem{definition}[theorem]{Definition}


\def\R{\mathbb{ R}}
\def\S{\mathbb{ S}}
\def\I{\mathbb{ I}}

\def\Scala{\texttt{Scala}}
\def\ScalaCheck{\texttt{ScalaCheck}}
\def\Haskell{\texttt{Haskell}}
\def\XML{\texttt{XML}}



\makeindex


\title{Pro Scala: Monadic Design Patterns for the Web}

\author{L.G. Meredith  \\
{\small\em \copyright \  Draft date \today }}

 \date{ }
\begin{document}
\lstset{language=Haskell}
\maketitle
 \addcontentsline{toc}{chapter}{Contents}
\pagenumbering{roman}
\tableofcontents
\listoffigures
\listoftables
\chapter*{Preface}\normalsize
  \addcontentsline{toc}{chapter}{Preface}
\pagestyle{plain}
% The book root file {\tt bookex.tex} gives a basic example of how to
% use \LaTeX \ for preparation of a book. Note that all
% \LaTeX \ commands begin with a
% backslash.

% Each
% Chapter, Appendix and the Index is made as a {\tt *.tex} file and is
% called in by the {\tt include} command---thus {\tt ch1.tex} is
% the name here of the file containing Chapter~1. The inclusion of any
% particular file can be suppressed by prefixing the line by a
% percent sign.


%  Do not put an {\tt end{document}} command at the end of chapter files;
% just one such command is needed at the end of the book.

% Note the tag used to make an index entry. You may need to consult Lamport's
% book~\cite{lamport} for details of the procedure to make the index input
% file; \LaTeX \ will create a pre-index by listing all the tagged
% items in the file {\tt bookex.idx} then you edit this into
% a {\tt theindex} environment, as {\tt index.tex}.

The book you hold in your hands, Dear Reader, is not at all what you expected...



\pagestyle{headings}
\pagenumbering{arabic}

%ch.tex


\chapter{Putting it all together}
\begin{center}
{\small\em The application as a whole}
\end{center}

\input{chapters/nine/putting-it-all-together}
\section{Our web application end-to-end}

TBD
\section{Deploying our application}

\subsection{Why we are not deploying on GAE}
\section{From one web application to web framework}

TBD

% \section{Existence problems}
% We begin with some metamathematics.
% All problems about the existence of maps can be cast into one of the
% following two forms, which are in a sense mutually dual.

% \noindent
% {\bf The Extension Problem}\index{extension problem} \    %%% NB index entry tag
% Given an inclusion $ A \stackrel{i}{\hookrightarrow} X $, and a map
% $ A \stackrel{f}{\rightarrow} Y $,
% does there exist a map $f^{\dagger}:X\to Y$ such that
% $f^{\dagger}$ agrees with $f$ on $A$?

% Here the appropriate source category for maps should be clear from the
% context and, moreover, commutativity through a
% candidate $f^{\dagger}$ is precisely
% the restriction requirement; that is,
% $$f^{\dagger}   :  f^{\dagger}\circ i = f^{\dagger}|_A = f\,. $$
% If such an $f^{\dagger}$ exists\footnote{${}^{\dagger}$ suggests striving
% for perfection, crusading}, then it is called an {\bf
% extension}\index{extension!of a map|bi} of $f$ and is said to {\bf
% extend}\index{extend|bi} $f$. In any diagrams, the presence of
% a dotted arrow or an arrow carrying a ? indicates a pious hope, in no way
% begging the question of its existence. Note that we shall usually
% omit $\circ$ from composite maps.

% \noindent
% {\bf The Lifting Problem}\index{lifting problem} \
% Given a pair of maps $E \stackrel{p}{\rightarrow}B$ and $X \stackrel{f}
% {\rightarrow} B $,
% does there exist a map $f^{\circ} : X \to E$, with
% $pf^{\circ} = f  $?


% That {\em all\/} existence problems about maps are essentially of one
% type or
% the other from these two is seen as follows. Evidently, all existence problems
% are representable by triangular diagrams\index{triangular diagrams} and it
% is easily seen that there are only these six possibilities:
% \begin{center}\begin{picture}(300,70)  %augch2 75
% \put(5,60){\vector(1,0){30}}
% \put(55,60){\vector(1,0){30}}
% \put(135,60){\vector(-1,0){30}}
% \put(185,60){\vector(-1,0){30}}
% \put(235,60){\vector(-1,0){30}}
% \put(285,60){\vector(-1,0){30}}
% \put(0,55){\vector(0,-1){30}}
% \put(50,55){\vector(0,-1){30}}
% \put(100,25){\vector(0,1){30}}
% \put(150,25){\vector(0,1){30}}
% \put(200,55){\vector(0,-1){30}}
% \put(250,55){\vector(0,-1){30}}
% \put(28,33){\small ?}
% \put(78,33){\small ?}
% \put(128,33){\small ?}
% \put(178,33){\small ?}
% \put(228,33){\small ?}
% \put(278,33){\small ?}
% \put(10,3){\bf 1}
% \put(60,3){\bf 2}
% \put(110,3){\bf 3}
% \put(160,3){\bf 4}
% \put(210,3){\bf 5}
% \put(260,3){\bf 6}
% \put(35,55){\vector(-1,-1){30}}
% \put(155,25){\vector(1,1){30}}
% \put(135,55){\vector(-1,-1){30}}
% \put(55,25){\vector(1,1){30}}
% \put(235,55){\vector(-1,-1){30}}
% \put(255,25){\vector(1,1){30}}
% \end{picture}\end{center}



% \begin{figure}
% \begin{picture}(300,220)(0,0)
% \put(-20,-20){\resizebox{20 cm}{!}{\includegraphics{3dpdf}}}
% \put(260,-10){\resizebox{15 cm}{!}{\includegraphics{contpdf}}}
% \put(220,80){$\beta$}
% \put(400,-10){$N$}
% \put(260,170){$\beta$}
% \put(90,15){$N$}
% \end{picture}
% \caption{{\em The log-gamma family of densities with central mean
% $<N> \, = \frac{1}{2}$ as a surface and as a contour plot. }}
% \label{pdf}
% \end{figure}

\newpage

%ch.tex


\chapter{Putting it all together}
\begin{center}
{\small\em The application as a whole}
\end{center}

\input{chapters/nine/putting-it-all-together}
\section{Our web application end-to-end}

TBD
\section{Deploying our application}

\subsection{Why we are not deploying on GAE}
\section{From one web application to web framework}

TBD

% \section{Existence problems}
% We begin with some metamathematics.
% All problems about the existence of maps can be cast into one of the
% following two forms, which are in a sense mutually dual.

% \noindent
% {\bf The Extension Problem}\index{extension problem} \    %%% NB index entry tag
% Given an inclusion $ A \stackrel{i}{\hookrightarrow} X $, and a map
% $ A \stackrel{f}{\rightarrow} Y $,
% does there exist a map $f^{\dagger}:X\to Y$ such that
% $f^{\dagger}$ agrees with $f$ on $A$?

% Here the appropriate source category for maps should be clear from the
% context and, moreover, commutativity through a
% candidate $f^{\dagger}$ is precisely
% the restriction requirement; that is,
% $$f^{\dagger}   :  f^{\dagger}\circ i = f^{\dagger}|_A = f\,. $$
% If such an $f^{\dagger}$ exists\footnote{${}^{\dagger}$ suggests striving
% for perfection, crusading}, then it is called an {\bf
% extension}\index{extension!of a map|bi} of $f$ and is said to {\bf
% extend}\index{extend|bi} $f$. In any diagrams, the presence of
% a dotted arrow or an arrow carrying a ? indicates a pious hope, in no way
% begging the question of its existence. Note that we shall usually
% omit $\circ$ from composite maps.

% \noindent
% {\bf The Lifting Problem}\index{lifting problem} \
% Given a pair of maps $E \stackrel{p}{\rightarrow}B$ and $X \stackrel{f}
% {\rightarrow} B $,
% does there exist a map $f^{\circ} : X \to E$, with
% $pf^{\circ} = f  $?


% That {\em all\/} existence problems about maps are essentially of one
% type or
% the other from these two is seen as follows. Evidently, all existence problems
% are representable by triangular diagrams\index{triangular diagrams} and it
% is easily seen that there are only these six possibilities:
% \begin{center}\begin{picture}(300,70)  %augch2 75
% \put(5,60){\vector(1,0){30}}
% \put(55,60){\vector(1,0){30}}
% \put(135,60){\vector(-1,0){30}}
% \put(185,60){\vector(-1,0){30}}
% \put(235,60){\vector(-1,0){30}}
% \put(285,60){\vector(-1,0){30}}
% \put(0,55){\vector(0,-1){30}}
% \put(50,55){\vector(0,-1){30}}
% \put(100,25){\vector(0,1){30}}
% \put(150,25){\vector(0,1){30}}
% \put(200,55){\vector(0,-1){30}}
% \put(250,55){\vector(0,-1){30}}
% \put(28,33){\small ?}
% \put(78,33){\small ?}
% \put(128,33){\small ?}
% \put(178,33){\small ?}
% \put(228,33){\small ?}
% \put(278,33){\small ?}
% \put(10,3){\bf 1}
% \put(60,3){\bf 2}
% \put(110,3){\bf 3}
% \put(160,3){\bf 4}
% \put(210,3){\bf 5}
% \put(260,3){\bf 6}
% \put(35,55){\vector(-1,-1){30}}
% \put(155,25){\vector(1,1){30}}
% \put(135,55){\vector(-1,-1){30}}
% \put(55,25){\vector(1,1){30}}
% \put(235,55){\vector(-1,-1){30}}
% \put(255,25){\vector(1,1){30}}
% \end{picture}\end{center}



% \begin{figure}
% \begin{picture}(300,220)(0,0)
% \put(-20,-20){\resizebox{20 cm}{!}{\includegraphics{3dpdf}}}
% \put(260,-10){\resizebox{15 cm}{!}{\includegraphics{contpdf}}}
% \put(220,80){$\beta$}
% \put(400,-10){$N$}
% \put(260,170){$\beta$}
% \put(90,15){$N$}
% \end{picture}
% \caption{{\em The log-gamma family of densities with central mean
% $<N> \, = \frac{1}{2}$ as a surface and as a contour plot. }}
% \label{pdf}
% \end{figure}

\newpage

%ch.tex


\chapter{Putting it all together}
\begin{center}
{\small\em The application as a whole}
\end{center}

\input{chapters/nine/putting-it-all-together}
\section{Our web application end-to-end}

TBD
\section{Deploying our application}

\subsection{Why we are not deploying on GAE}
\section{From one web application to web framework}

TBD

% \section{Existence problems}
% We begin with some metamathematics.
% All problems about the existence of maps can be cast into one of the
% following two forms, which are in a sense mutually dual.

% \noindent
% {\bf The Extension Problem}\index{extension problem} \    %%% NB index entry tag
% Given an inclusion $ A \stackrel{i}{\hookrightarrow} X $, and a map
% $ A \stackrel{f}{\rightarrow} Y $,
% does there exist a map $f^{\dagger}:X\to Y$ such that
% $f^{\dagger}$ agrees with $f$ on $A$?

% Here the appropriate source category for maps should be clear from the
% context and, moreover, commutativity through a
% candidate $f^{\dagger}$ is precisely
% the restriction requirement; that is,
% $$f^{\dagger}   :  f^{\dagger}\circ i = f^{\dagger}|_A = f\,. $$
% If such an $f^{\dagger}$ exists\footnote{${}^{\dagger}$ suggests striving
% for perfection, crusading}, then it is called an {\bf
% extension}\index{extension!of a map|bi} of $f$ and is said to {\bf
% extend}\index{extend|bi} $f$. In any diagrams, the presence of
% a dotted arrow or an arrow carrying a ? indicates a pious hope, in no way
% begging the question of its existence. Note that we shall usually
% omit $\circ$ from composite maps.

% \noindent
% {\bf The Lifting Problem}\index{lifting problem} \
% Given a pair of maps $E \stackrel{p}{\rightarrow}B$ and $X \stackrel{f}
% {\rightarrow} B $,
% does there exist a map $f^{\circ} : X \to E$, with
% $pf^{\circ} = f  $?


% That {\em all\/} existence problems about maps are essentially of one
% type or
% the other from these two is seen as follows. Evidently, all existence problems
% are representable by triangular diagrams\index{triangular diagrams} and it
% is easily seen that there are only these six possibilities:
% \begin{center}\begin{picture}(300,70)  %augch2 75
% \put(5,60){\vector(1,0){30}}
% \put(55,60){\vector(1,0){30}}
% \put(135,60){\vector(-1,0){30}}
% \put(185,60){\vector(-1,0){30}}
% \put(235,60){\vector(-1,0){30}}
% \put(285,60){\vector(-1,0){30}}
% \put(0,55){\vector(0,-1){30}}
% \put(50,55){\vector(0,-1){30}}
% \put(100,25){\vector(0,1){30}}
% \put(150,25){\vector(0,1){30}}
% \put(200,55){\vector(0,-1){30}}
% \put(250,55){\vector(0,-1){30}}
% \put(28,33){\small ?}
% \put(78,33){\small ?}
% \put(128,33){\small ?}
% \put(178,33){\small ?}
% \put(228,33){\small ?}
% \put(278,33){\small ?}
% \put(10,3){\bf 1}
% \put(60,3){\bf 2}
% \put(110,3){\bf 3}
% \put(160,3){\bf 4}
% \put(210,3){\bf 5}
% \put(260,3){\bf 6}
% \put(35,55){\vector(-1,-1){30}}
% \put(155,25){\vector(1,1){30}}
% \put(135,55){\vector(-1,-1){30}}
% \put(55,25){\vector(1,1){30}}
% \put(235,55){\vector(-1,-1){30}}
% \put(255,25){\vector(1,1){30}}
% \end{picture}\end{center}



% \begin{figure}
% \begin{picture}(300,220)(0,0)
% \put(-20,-20){\resizebox{20 cm}{!}{\includegraphics{3dpdf}}}
% \put(260,-10){\resizebox{15 cm}{!}{\includegraphics{contpdf}}}
% \put(220,80){$\beta$}
% \put(400,-10){$N$}
% \put(260,170){$\beta$}
% \put(90,15){$N$}
% \end{picture}
% \caption{{\em The log-gamma family of densities with central mean
% $<N> \, = \frac{1}{2}$ as a surface and as a contour plot. }}
% \label{pdf}
% \end{figure}

\newpage

%ch.tex


\chapter{Putting it all together}
\begin{center}
{\small\em The application as a whole}
\end{center}

\input{chapters/nine/putting-it-all-together}
\section{Our web application end-to-end}

TBD
\section{Deploying our application}

\subsection{Why we are not deploying on GAE}
\section{From one web application to web framework}

TBD

% \section{Existence problems}
% We begin with some metamathematics.
% All problems about the existence of maps can be cast into one of the
% following two forms, which are in a sense mutually dual.

% \noindent
% {\bf The Extension Problem}\index{extension problem} \    %%% NB index entry tag
% Given an inclusion $ A \stackrel{i}{\hookrightarrow} X $, and a map
% $ A \stackrel{f}{\rightarrow} Y $,
% does there exist a map $f^{\dagger}:X\to Y$ such that
% $f^{\dagger}$ agrees with $f$ on $A$?

% Here the appropriate source category for maps should be clear from the
% context and, moreover, commutativity through a
% candidate $f^{\dagger}$ is precisely
% the restriction requirement; that is,
% $$f^{\dagger}   :  f^{\dagger}\circ i = f^{\dagger}|_A = f\,. $$
% If such an $f^{\dagger}$ exists\footnote{${}^{\dagger}$ suggests striving
% for perfection, crusading}, then it is called an {\bf
% extension}\index{extension!of a map|bi} of $f$ and is said to {\bf
% extend}\index{extend|bi} $f$. In any diagrams, the presence of
% a dotted arrow or an arrow carrying a ? indicates a pious hope, in no way
% begging the question of its existence. Note that we shall usually
% omit $\circ$ from composite maps.

% \noindent
% {\bf The Lifting Problem}\index{lifting problem} \
% Given a pair of maps $E \stackrel{p}{\rightarrow}B$ and $X \stackrel{f}
% {\rightarrow} B $,
% does there exist a map $f^{\circ} : X \to E$, with
% $pf^{\circ} = f  $?


% That {\em all\/} existence problems about maps are essentially of one
% type or
% the other from these two is seen as follows. Evidently, all existence problems
% are representable by triangular diagrams\index{triangular diagrams} and it
% is easily seen that there are only these six possibilities:
% \begin{center}\begin{picture}(300,70)  %augch2 75
% \put(5,60){\vector(1,0){30}}
% \put(55,60){\vector(1,0){30}}
% \put(135,60){\vector(-1,0){30}}
% \put(185,60){\vector(-1,0){30}}
% \put(235,60){\vector(-1,0){30}}
% \put(285,60){\vector(-1,0){30}}
% \put(0,55){\vector(0,-1){30}}
% \put(50,55){\vector(0,-1){30}}
% \put(100,25){\vector(0,1){30}}
% \put(150,25){\vector(0,1){30}}
% \put(200,55){\vector(0,-1){30}}
% \put(250,55){\vector(0,-1){30}}
% \put(28,33){\small ?}
% \put(78,33){\small ?}
% \put(128,33){\small ?}
% \put(178,33){\small ?}
% \put(228,33){\small ?}
% \put(278,33){\small ?}
% \put(10,3){\bf 1}
% \put(60,3){\bf 2}
% \put(110,3){\bf 3}
% \put(160,3){\bf 4}
% \put(210,3){\bf 5}
% \put(260,3){\bf 6}
% \put(35,55){\vector(-1,-1){30}}
% \put(155,25){\vector(1,1){30}}
% \put(135,55){\vector(-1,-1){30}}
% \put(55,25){\vector(1,1){30}}
% \put(235,55){\vector(-1,-1){30}}
% \put(255,25){\vector(1,1){30}}
% \end{picture}\end{center}



% \begin{figure}
% \begin{picture}(300,220)(0,0)
% \put(-20,-20){\resizebox{20 cm}{!}{\includegraphics{3dpdf}}}
% \put(260,-10){\resizebox{15 cm}{!}{\includegraphics{contpdf}}}
% \put(220,80){$\beta$}
% \put(400,-10){$N$}
% \put(260,170){$\beta$}
% \put(90,15){$N$}
% \end{picture}
% \caption{{\em The log-gamma family of densities with central mean
% $<N> \, = \frac{1}{2}$ as a surface and as a contour plot. }}
% \label{pdf}
% \end{figure}

\newpage

%ch.tex


\chapter{Putting it all together}
\begin{center}
{\small\em The application as a whole}
\end{center}

\input{chapters/nine/putting-it-all-together}
\section{Our web application end-to-end}

TBD
\section{Deploying our application}

\subsection{Why we are not deploying on GAE}
\section{From one web application to web framework}

TBD

% \section{Existence problems}
% We begin with some metamathematics.
% All problems about the existence of maps can be cast into one of the
% following two forms, which are in a sense mutually dual.

% \noindent
% {\bf The Extension Problem}\index{extension problem} \    %%% NB index entry tag
% Given an inclusion $ A \stackrel{i}{\hookrightarrow} X $, and a map
% $ A \stackrel{f}{\rightarrow} Y $,
% does there exist a map $f^{\dagger}:X\to Y$ such that
% $f^{\dagger}$ agrees with $f$ on $A$?

% Here the appropriate source category for maps should be clear from the
% context and, moreover, commutativity through a
% candidate $f^{\dagger}$ is precisely
% the restriction requirement; that is,
% $$f^{\dagger}   :  f^{\dagger}\circ i = f^{\dagger}|_A = f\,. $$
% If such an $f^{\dagger}$ exists\footnote{${}^{\dagger}$ suggests striving
% for perfection, crusading}, then it is called an {\bf
% extension}\index{extension!of a map|bi} of $f$ and is said to {\bf
% extend}\index{extend|bi} $f$. In any diagrams, the presence of
% a dotted arrow or an arrow carrying a ? indicates a pious hope, in no way
% begging the question of its existence. Note that we shall usually
% omit $\circ$ from composite maps.

% \noindent
% {\bf The Lifting Problem}\index{lifting problem} \
% Given a pair of maps $E \stackrel{p}{\rightarrow}B$ and $X \stackrel{f}
% {\rightarrow} B $,
% does there exist a map $f^{\circ} : X \to E$, with
% $pf^{\circ} = f  $?


% That {\em all\/} existence problems about maps are essentially of one
% type or
% the other from these two is seen as follows. Evidently, all existence problems
% are representable by triangular diagrams\index{triangular diagrams} and it
% is easily seen that there are only these six possibilities:
% \begin{center}\begin{picture}(300,70)  %augch2 75
% \put(5,60){\vector(1,0){30}}
% \put(55,60){\vector(1,0){30}}
% \put(135,60){\vector(-1,0){30}}
% \put(185,60){\vector(-1,0){30}}
% \put(235,60){\vector(-1,0){30}}
% \put(285,60){\vector(-1,0){30}}
% \put(0,55){\vector(0,-1){30}}
% \put(50,55){\vector(0,-1){30}}
% \put(100,25){\vector(0,1){30}}
% \put(150,25){\vector(0,1){30}}
% \put(200,55){\vector(0,-1){30}}
% \put(250,55){\vector(0,-1){30}}
% \put(28,33){\small ?}
% \put(78,33){\small ?}
% \put(128,33){\small ?}
% \put(178,33){\small ?}
% \put(228,33){\small ?}
% \put(278,33){\small ?}
% \put(10,3){\bf 1}
% \put(60,3){\bf 2}
% \put(110,3){\bf 3}
% \put(160,3){\bf 4}
% \put(210,3){\bf 5}
% \put(260,3){\bf 6}
% \put(35,55){\vector(-1,-1){30}}
% \put(155,25){\vector(1,1){30}}
% \put(135,55){\vector(-1,-1){30}}
% \put(55,25){\vector(1,1){30}}
% \put(235,55){\vector(-1,-1){30}}
% \put(255,25){\vector(1,1){30}}
% \end{picture}\end{center}



% \begin{figure}
% \begin{picture}(300,220)(0,0)
% \put(-20,-20){\resizebox{20 cm}{!}{\includegraphics{3dpdf}}}
% \put(260,-10){\resizebox{15 cm}{!}{\includegraphics{contpdf}}}
% \put(220,80){$\beta$}
% \put(400,-10){$N$}
% \put(260,170){$\beta$}
% \put(90,15){$N$}
% \end{picture}
% \caption{{\em The log-gamma family of densities with central mean
% $<N> \, = \frac{1}{2}$ as a surface and as a contour plot. }}
% \label{pdf}
% \end{figure}

\newpage

%ch.tex


\chapter{Putting it all together}
\begin{center}
{\small\em The application as a whole}
\end{center}

\input{chapters/nine/putting-it-all-together}
\section{Our web application end-to-end}

TBD
\section{Deploying our application}

\subsection{Why we are not deploying on GAE}
\section{From one web application to web framework}

TBD

% \section{Existence problems}
% We begin with some metamathematics.
% All problems about the existence of maps can be cast into one of the
% following two forms, which are in a sense mutually dual.

% \noindent
% {\bf The Extension Problem}\index{extension problem} \    %%% NB index entry tag
% Given an inclusion $ A \stackrel{i}{\hookrightarrow} X $, and a map
% $ A \stackrel{f}{\rightarrow} Y $,
% does there exist a map $f^{\dagger}:X\to Y$ such that
% $f^{\dagger}$ agrees with $f$ on $A$?

% Here the appropriate source category for maps should be clear from the
% context and, moreover, commutativity through a
% candidate $f^{\dagger}$ is precisely
% the restriction requirement; that is,
% $$f^{\dagger}   :  f^{\dagger}\circ i = f^{\dagger}|_A = f\,. $$
% If such an $f^{\dagger}$ exists\footnote{${}^{\dagger}$ suggests striving
% for perfection, crusading}, then it is called an {\bf
% extension}\index{extension!of a map|bi} of $f$ and is said to {\bf
% extend}\index{extend|bi} $f$. In any diagrams, the presence of
% a dotted arrow or an arrow carrying a ? indicates a pious hope, in no way
% begging the question of its existence. Note that we shall usually
% omit $\circ$ from composite maps.

% \noindent
% {\bf The Lifting Problem}\index{lifting problem} \
% Given a pair of maps $E \stackrel{p}{\rightarrow}B$ and $X \stackrel{f}
% {\rightarrow} B $,
% does there exist a map $f^{\circ} : X \to E$, with
% $pf^{\circ} = f  $?


% That {\em all\/} existence problems about maps are essentially of one
% type or
% the other from these two is seen as follows. Evidently, all existence problems
% are representable by triangular diagrams\index{triangular diagrams} and it
% is easily seen that there are only these six possibilities:
% \begin{center}\begin{picture}(300,70)  %augch2 75
% \put(5,60){\vector(1,0){30}}
% \put(55,60){\vector(1,0){30}}
% \put(135,60){\vector(-1,0){30}}
% \put(185,60){\vector(-1,0){30}}
% \put(235,60){\vector(-1,0){30}}
% \put(285,60){\vector(-1,0){30}}
% \put(0,55){\vector(0,-1){30}}
% \put(50,55){\vector(0,-1){30}}
% \put(100,25){\vector(0,1){30}}
% \put(150,25){\vector(0,1){30}}
% \put(200,55){\vector(0,-1){30}}
% \put(250,55){\vector(0,-1){30}}
% \put(28,33){\small ?}
% \put(78,33){\small ?}
% \put(128,33){\small ?}
% \put(178,33){\small ?}
% \put(228,33){\small ?}
% \put(278,33){\small ?}
% \put(10,3){\bf 1}
% \put(60,3){\bf 2}
% \put(110,3){\bf 3}
% \put(160,3){\bf 4}
% \put(210,3){\bf 5}
% \put(260,3){\bf 6}
% \put(35,55){\vector(-1,-1){30}}
% \put(155,25){\vector(1,1){30}}
% \put(135,55){\vector(-1,-1){30}}
% \put(55,25){\vector(1,1){30}}
% \put(235,55){\vector(-1,-1){30}}
% \put(255,25){\vector(1,1){30}}
% \end{picture}\end{center}



% \begin{figure}
% \begin{picture}(300,220)(0,0)
% \put(-20,-20){\resizebox{20 cm}{!}{\includegraphics{3dpdf}}}
% \put(260,-10){\resizebox{15 cm}{!}{\includegraphics{contpdf}}}
% \put(220,80){$\beta$}
% \put(400,-10){$N$}
% \put(260,170){$\beta$}
% \put(90,15){$N$}
% \end{picture}
% \caption{{\em The log-gamma family of densities with central mean
% $<N> \, = \frac{1}{2}$ as a surface and as a contour plot. }}
% \label{pdf}
% \end{figure}

\newpage

%ch.tex


\chapter{Putting it all together}
\begin{center}
{\small\em The application as a whole}
\end{center}

\input{chapters/nine/putting-it-all-together}
\section{Our web application end-to-end}

TBD
\section{Deploying our application}

\subsection{Why we are not deploying on GAE}
\section{From one web application to web framework}

TBD

% \section{Existence problems}
% We begin with some metamathematics.
% All problems about the existence of maps can be cast into one of the
% following two forms, which are in a sense mutually dual.

% \noindent
% {\bf The Extension Problem}\index{extension problem} \    %%% NB index entry tag
% Given an inclusion $ A \stackrel{i}{\hookrightarrow} X $, and a map
% $ A \stackrel{f}{\rightarrow} Y $,
% does there exist a map $f^{\dagger}:X\to Y$ such that
% $f^{\dagger}$ agrees with $f$ on $A$?

% Here the appropriate source category for maps should be clear from the
% context and, moreover, commutativity through a
% candidate $f^{\dagger}$ is precisely
% the restriction requirement; that is,
% $$f^{\dagger}   :  f^{\dagger}\circ i = f^{\dagger}|_A = f\,. $$
% If such an $f^{\dagger}$ exists\footnote{${}^{\dagger}$ suggests striving
% for perfection, crusading}, then it is called an {\bf
% extension}\index{extension!of a map|bi} of $f$ and is said to {\bf
% extend}\index{extend|bi} $f$. In any diagrams, the presence of
% a dotted arrow or an arrow carrying a ? indicates a pious hope, in no way
% begging the question of its existence. Note that we shall usually
% omit $\circ$ from composite maps.

% \noindent
% {\bf The Lifting Problem}\index{lifting problem} \
% Given a pair of maps $E \stackrel{p}{\rightarrow}B$ and $X \stackrel{f}
% {\rightarrow} B $,
% does there exist a map $f^{\circ} : X \to E$, with
% $pf^{\circ} = f  $?


% That {\em all\/} existence problems about maps are essentially of one
% type or
% the other from these two is seen as follows. Evidently, all existence problems
% are representable by triangular diagrams\index{triangular diagrams} and it
% is easily seen that there are only these six possibilities:
% \begin{center}\begin{picture}(300,70)  %augch2 75
% \put(5,60){\vector(1,0){30}}
% \put(55,60){\vector(1,0){30}}
% \put(135,60){\vector(-1,0){30}}
% \put(185,60){\vector(-1,0){30}}
% \put(235,60){\vector(-1,0){30}}
% \put(285,60){\vector(-1,0){30}}
% \put(0,55){\vector(0,-1){30}}
% \put(50,55){\vector(0,-1){30}}
% \put(100,25){\vector(0,1){30}}
% \put(150,25){\vector(0,1){30}}
% \put(200,55){\vector(0,-1){30}}
% \put(250,55){\vector(0,-1){30}}
% \put(28,33){\small ?}
% \put(78,33){\small ?}
% \put(128,33){\small ?}
% \put(178,33){\small ?}
% \put(228,33){\small ?}
% \put(278,33){\small ?}
% \put(10,3){\bf 1}
% \put(60,3){\bf 2}
% \put(110,3){\bf 3}
% \put(160,3){\bf 4}
% \put(210,3){\bf 5}
% \put(260,3){\bf 6}
% \put(35,55){\vector(-1,-1){30}}
% \put(155,25){\vector(1,1){30}}
% \put(135,55){\vector(-1,-1){30}}
% \put(55,25){\vector(1,1){30}}
% \put(235,55){\vector(-1,-1){30}}
% \put(255,25){\vector(1,1){30}}
% \end{picture}\end{center}



% \begin{figure}
% \begin{picture}(300,220)(0,0)
% \put(-20,-20){\resizebox{20 cm}{!}{\includegraphics{3dpdf}}}
% \put(260,-10){\resizebox{15 cm}{!}{\includegraphics{contpdf}}}
% \put(220,80){$\beta$}
% \put(400,-10){$N$}
% \put(260,170){$\beta$}
% \put(90,15){$N$}
% \end{picture}
% \caption{{\em The log-gamma family of densities with central mean
% $<N> \, = \frac{1}{2}$ as a surface and as a contour plot. }}
% \label{pdf}
% \end{figure}

\newpage

%ch.tex


\chapter{Putting it all together}
\begin{center}
{\small\em The application as a whole}
\end{center}

\input{chapters/nine/putting-it-all-together}
\section{Our web application end-to-end}

TBD
\section{Deploying our application}

\subsection{Why we are not deploying on GAE}
\section{From one web application to web framework}

TBD

% \section{Existence problems}
% We begin with some metamathematics.
% All problems about the existence of maps can be cast into one of the
% following two forms, which are in a sense mutually dual.

% \noindent
% {\bf The Extension Problem}\index{extension problem} \    %%% NB index entry tag
% Given an inclusion $ A \stackrel{i}{\hookrightarrow} X $, and a map
% $ A \stackrel{f}{\rightarrow} Y $,
% does there exist a map $f^{\dagger}:X\to Y$ such that
% $f^{\dagger}$ agrees with $f$ on $A$?

% Here the appropriate source category for maps should be clear from the
% context and, moreover, commutativity through a
% candidate $f^{\dagger}$ is precisely
% the restriction requirement; that is,
% $$f^{\dagger}   :  f^{\dagger}\circ i = f^{\dagger}|_A = f\,. $$
% If such an $f^{\dagger}$ exists\footnote{${}^{\dagger}$ suggests striving
% for perfection, crusading}, then it is called an {\bf
% extension}\index{extension!of a map|bi} of $f$ and is said to {\bf
% extend}\index{extend|bi} $f$. In any diagrams, the presence of
% a dotted arrow or an arrow carrying a ? indicates a pious hope, in no way
% begging the question of its existence. Note that we shall usually
% omit $\circ$ from composite maps.

% \noindent
% {\bf The Lifting Problem}\index{lifting problem} \
% Given a pair of maps $E \stackrel{p}{\rightarrow}B$ and $X \stackrel{f}
% {\rightarrow} B $,
% does there exist a map $f^{\circ} : X \to E$, with
% $pf^{\circ} = f  $?


% That {\em all\/} existence problems about maps are essentially of one
% type or
% the other from these two is seen as follows. Evidently, all existence problems
% are representable by triangular diagrams\index{triangular diagrams} and it
% is easily seen that there are only these six possibilities:
% \begin{center}\begin{picture}(300,70)  %augch2 75
% \put(5,60){\vector(1,0){30}}
% \put(55,60){\vector(1,0){30}}
% \put(135,60){\vector(-1,0){30}}
% \put(185,60){\vector(-1,0){30}}
% \put(235,60){\vector(-1,0){30}}
% \put(285,60){\vector(-1,0){30}}
% \put(0,55){\vector(0,-1){30}}
% \put(50,55){\vector(0,-1){30}}
% \put(100,25){\vector(0,1){30}}
% \put(150,25){\vector(0,1){30}}
% \put(200,55){\vector(0,-1){30}}
% \put(250,55){\vector(0,-1){30}}
% \put(28,33){\small ?}
% \put(78,33){\small ?}
% \put(128,33){\small ?}
% \put(178,33){\small ?}
% \put(228,33){\small ?}
% \put(278,33){\small ?}
% \put(10,3){\bf 1}
% \put(60,3){\bf 2}
% \put(110,3){\bf 3}
% \put(160,3){\bf 4}
% \put(210,3){\bf 5}
% \put(260,3){\bf 6}
% \put(35,55){\vector(-1,-1){30}}
% \put(155,25){\vector(1,1){30}}
% \put(135,55){\vector(-1,-1){30}}
% \put(55,25){\vector(1,1){30}}
% \put(235,55){\vector(-1,-1){30}}
% \put(255,25){\vector(1,1){30}}
% \end{picture}\end{center}



% \begin{figure}
% \begin{picture}(300,220)(0,0)
% \put(-20,-20){\resizebox{20 cm}{!}{\includegraphics{3dpdf}}}
% \put(260,-10){\resizebox{15 cm}{!}{\includegraphics{contpdf}}}
% \put(220,80){$\beta$}
% \put(400,-10){$N$}
% \put(260,170){$\beta$}
% \put(90,15){$N$}
% \end{picture}
% \caption{{\em The log-gamma family of densities with central mean
% $<N> \, = \frac{1}{2}$ as a surface and as a contour plot. }}
% \label{pdf}
% \end{figure}

\newpage

%ch.tex


\chapter{Putting it all together}
\begin{center}
{\small\em The application as a whole}
\end{center}

\input{chapters/nine/putting-it-all-together}
\section{Our web application end-to-end}

TBD
\section{Deploying our application}

\subsection{Why we are not deploying on GAE}
\section{From one web application to web framework}

TBD

% \section{Existence problems}
% We begin with some metamathematics.
% All problems about the existence of maps can be cast into one of the
% following two forms, which are in a sense mutually dual.

% \noindent
% {\bf The Extension Problem}\index{extension problem} \    %%% NB index entry tag
% Given an inclusion $ A \stackrel{i}{\hookrightarrow} X $, and a map
% $ A \stackrel{f}{\rightarrow} Y $,
% does there exist a map $f^{\dagger}:X\to Y$ such that
% $f^{\dagger}$ agrees with $f$ on $A$?

% Here the appropriate source category for maps should be clear from the
% context and, moreover, commutativity through a
% candidate $f^{\dagger}$ is precisely
% the restriction requirement; that is,
% $$f^{\dagger}   :  f^{\dagger}\circ i = f^{\dagger}|_A = f\,. $$
% If such an $f^{\dagger}$ exists\footnote{${}^{\dagger}$ suggests striving
% for perfection, crusading}, then it is called an {\bf
% extension}\index{extension!of a map|bi} of $f$ and is said to {\bf
% extend}\index{extend|bi} $f$. In any diagrams, the presence of
% a dotted arrow or an arrow carrying a ? indicates a pious hope, in no way
% begging the question of its existence. Note that we shall usually
% omit $\circ$ from composite maps.

% \noindent
% {\bf The Lifting Problem}\index{lifting problem} \
% Given a pair of maps $E \stackrel{p}{\rightarrow}B$ and $X \stackrel{f}
% {\rightarrow} B $,
% does there exist a map $f^{\circ} : X \to E$, with
% $pf^{\circ} = f  $?


% That {\em all\/} existence problems about maps are essentially of one
% type or
% the other from these two is seen as follows. Evidently, all existence problems
% are representable by triangular diagrams\index{triangular diagrams} and it
% is easily seen that there are only these six possibilities:
% \begin{center}\begin{picture}(300,70)  %augch2 75
% \put(5,60){\vector(1,0){30}}
% \put(55,60){\vector(1,0){30}}
% \put(135,60){\vector(-1,0){30}}
% \put(185,60){\vector(-1,0){30}}
% \put(235,60){\vector(-1,0){30}}
% \put(285,60){\vector(-1,0){30}}
% \put(0,55){\vector(0,-1){30}}
% \put(50,55){\vector(0,-1){30}}
% \put(100,25){\vector(0,1){30}}
% \put(150,25){\vector(0,1){30}}
% \put(200,55){\vector(0,-1){30}}
% \put(250,55){\vector(0,-1){30}}
% \put(28,33){\small ?}
% \put(78,33){\small ?}
% \put(128,33){\small ?}
% \put(178,33){\small ?}
% \put(228,33){\small ?}
% \put(278,33){\small ?}
% \put(10,3){\bf 1}
% \put(60,3){\bf 2}
% \put(110,3){\bf 3}
% \put(160,3){\bf 4}
% \put(210,3){\bf 5}
% \put(260,3){\bf 6}
% \put(35,55){\vector(-1,-1){30}}
% \put(155,25){\vector(1,1){30}}
% \put(135,55){\vector(-1,-1){30}}
% \put(55,25){\vector(1,1){30}}
% \put(235,55){\vector(-1,-1){30}}
% \put(255,25){\vector(1,1){30}}
% \end{picture}\end{center}



% \begin{figure}
% \begin{picture}(300,220)(0,0)
% \put(-20,-20){\resizebox{20 cm}{!}{\includegraphics{3dpdf}}}
% \put(260,-10){\resizebox{15 cm}{!}{\includegraphics{contpdf}}}
% \put(220,80){$\beta$}
% \put(400,-10){$N$}
% \put(260,170){$\beta$}
% \put(90,15){$N$}
% \end{picture}
% \caption{{\em The log-gamma family of densities with central mean
% $<N> \, = \frac{1}{2}$ as a surface and as a contour plot. }}
% \label{pdf}
% \end{figure}

\newpage

%ch.tex


\chapter{Putting it all together}
\begin{center}
{\small\em The application as a whole}
\end{center}

\input{chapters/nine/putting-it-all-together}
\section{Our web application end-to-end}

TBD
\section{Deploying our application}

\subsection{Why we are not deploying on GAE}
\section{From one web application to web framework}

TBD

% \section{Existence problems}
% We begin with some metamathematics.
% All problems about the existence of maps can be cast into one of the
% following two forms, which are in a sense mutually dual.

% \noindent
% {\bf The Extension Problem}\index{extension problem} \    %%% NB index entry tag
% Given an inclusion $ A \stackrel{i}{\hookrightarrow} X $, and a map
% $ A \stackrel{f}{\rightarrow} Y $,
% does there exist a map $f^{\dagger}:X\to Y$ such that
% $f^{\dagger}$ agrees with $f$ on $A$?

% Here the appropriate source category for maps should be clear from the
% context and, moreover, commutativity through a
% candidate $f^{\dagger}$ is precisely
% the restriction requirement; that is,
% $$f^{\dagger}   :  f^{\dagger}\circ i = f^{\dagger}|_A = f\,. $$
% If such an $f^{\dagger}$ exists\footnote{${}^{\dagger}$ suggests striving
% for perfection, crusading}, then it is called an {\bf
% extension}\index{extension!of a map|bi} of $f$ and is said to {\bf
% extend}\index{extend|bi} $f$. In any diagrams, the presence of
% a dotted arrow or an arrow carrying a ? indicates a pious hope, in no way
% begging the question of its existence. Note that we shall usually
% omit $\circ$ from composite maps.

% \noindent
% {\bf The Lifting Problem}\index{lifting problem} \
% Given a pair of maps $E \stackrel{p}{\rightarrow}B$ and $X \stackrel{f}
% {\rightarrow} B $,
% does there exist a map $f^{\circ} : X \to E$, with
% $pf^{\circ} = f  $?


% That {\em all\/} existence problems about maps are essentially of one
% type or
% the other from these two is seen as follows. Evidently, all existence problems
% are representable by triangular diagrams\index{triangular diagrams} and it
% is easily seen that there are only these six possibilities:
% \begin{center}\begin{picture}(300,70)  %augch2 75
% \put(5,60){\vector(1,0){30}}
% \put(55,60){\vector(1,0){30}}
% \put(135,60){\vector(-1,0){30}}
% \put(185,60){\vector(-1,0){30}}
% \put(235,60){\vector(-1,0){30}}
% \put(285,60){\vector(-1,0){30}}
% \put(0,55){\vector(0,-1){30}}
% \put(50,55){\vector(0,-1){30}}
% \put(100,25){\vector(0,1){30}}
% \put(150,25){\vector(0,1){30}}
% \put(200,55){\vector(0,-1){30}}
% \put(250,55){\vector(0,-1){30}}
% \put(28,33){\small ?}
% \put(78,33){\small ?}
% \put(128,33){\small ?}
% \put(178,33){\small ?}
% \put(228,33){\small ?}
% \put(278,33){\small ?}
% \put(10,3){\bf 1}
% \put(60,3){\bf 2}
% \put(110,3){\bf 3}
% \put(160,3){\bf 4}
% \put(210,3){\bf 5}
% \put(260,3){\bf 6}
% \put(35,55){\vector(-1,-1){30}}
% \put(155,25){\vector(1,1){30}}
% \put(135,55){\vector(-1,-1){30}}
% \put(55,25){\vector(1,1){30}}
% \put(235,55){\vector(-1,-1){30}}
% \put(255,25){\vector(1,1){30}}
% \end{picture}\end{center}



% \begin{figure}
% \begin{picture}(300,220)(0,0)
% \put(-20,-20){\resizebox{20 cm}{!}{\includegraphics{3dpdf}}}
% \put(260,-10){\resizebox{15 cm}{!}{\includegraphics{contpdf}}}
% \put(220,80){$\beta$}
% \put(400,-10){$N$}
% \put(260,170){$\beta$}
% \put(90,15){$N$}
% \end{picture}
% \caption{{\em The log-gamma family of densities with central mean
% $<N> \, = \frac{1}{2}$ as a surface and as a contour plot. }}
% \label{pdf}
% \end{figure}

\newpage



\documentclass[12pt,leqno]{book}
\usepackage{amsmath,amssymb,amsfonts} % Typical maths resource packages
\usepackage{graphicx}                 % Packages to allow inclusion of graphics
\usepackage{color}                    % For creating coloured text and background
\usepackage{hyperref}                 % For creating hyperlinks in cross references
\usepackage{makeidx}                  % For indexing
\usepackage{listings}                 % For code listing
\usepackage{mathpartir}               % For grammars, rules, etc
\usepackage{bcprules}                 % For other kinds of rules
\usepackage{diagrams}                 % For commutative diagrams

\lstloadlanguages{Scala,Java,Haskell,XML,bash,HTML,SQL}

\parindent 1cm
\parskip 0.2cm
\topmargin 0.2cm
\oddsidemargin 1cm
\evensidemargin 0.5cm
\textwidth 15cm
\textheight 21cm

\newtheorem{theorem}{Theorem}[section]
\newtheorem{proposition}[theorem]{Proposition}
\newtheorem{corollary}[theorem]{Corollary}
\newtheorem{lemma}[theorem]{Lemma}
\newtheorem{remark}[theorem]{Remark}
\newtheorem{definition}[theorem]{Definition}


\def\R{\mathbb{ R}}
\def\S{\mathbb{ S}}
\def\I{\mathbb{ I}}

\def\Scala{\texttt{Scala}}
\def\ScalaCheck{\texttt{ScalaCheck}}
\def\Haskell{\texttt{Haskell}}
\def\XML{\texttt{XML}}



\makeindex


\title{Pro Scala: Monadic Design Patterns for the Web}

\author{L.G. Meredith  \\
{\small\em \copyright \  Draft date \today }}

 \date{ }
\begin{document}
\lstset{language=Haskell}
\maketitle
 \addcontentsline{toc}{chapter}{Contents}
\pagenumbering{roman}
\tableofcontents
\listoffigures
\listoftables
\chapter*{Preface}\normalsize
  \addcontentsline{toc}{chapter}{Preface}
\pagestyle{plain}
% The book root file {\tt bookex.tex} gives a basic example of how to
% use \LaTeX \ for preparation of a book. Note that all
% \LaTeX \ commands begin with a
% backslash.

% Each
% Chapter, Appendix and the Index is made as a {\tt *.tex} file and is
% called in by the {\tt include} command---thus {\tt ch1.tex} is
% the name here of the file containing Chapter~1. The inclusion of any
% particular file can be suppressed by prefixing the line by a
% percent sign.


%  Do not put an {\tt end{document}} command at the end of chapter files;
% just one such command is needed at the end of the book.

% Note the tag used to make an index entry. You may need to consult Lamport's
% book~\cite{lamport} for details of the procedure to make the index input
% file; \LaTeX \ will create a pre-index by listing all the tagged
% items in the file {\tt bookex.idx} then you edit this into
% a {\tt theindex} environment, as {\tt index.tex}.

The book you hold in your hands, Dear Reader, is not at all what you expected...



\pagestyle{headings}
\pagenumbering{arabic}

%ch.tex


\chapter{Putting it all together}
\begin{center}
{\small\em The application as a whole}
\end{center}

\input{chapters/nine/putting-it-all-together}
\input{chapters/nine/end-to-end}
\input{chapters/nine/deployment}
\input{chapters/nine/application-to-framework}

% \section{Existence problems}
% We begin with some metamathematics.
% All problems about the existence of maps can be cast into one of the
% following two forms, which are in a sense mutually dual.

% \noindent
% {\bf The Extension Problem}\index{extension problem} \    %%% NB index entry tag
% Given an inclusion $ A \stackrel{i}{\hookrightarrow} X $, and a map
% $ A \stackrel{f}{\rightarrow} Y $,
% does there exist a map $f^{\dagger}:X\to Y$ such that
% $f^{\dagger}$ agrees with $f$ on $A$?

% Here the appropriate source category for maps should be clear from the
% context and, moreover, commutativity through a
% candidate $f^{\dagger}$ is precisely
% the restriction requirement; that is,
% $$f^{\dagger}   :  f^{\dagger}\circ i = f^{\dagger}|_A = f\,. $$
% If such an $f^{\dagger}$ exists\footnote{${}^{\dagger}$ suggests striving
% for perfection, crusading}, then it is called an {\bf
% extension}\index{extension!of a map|bi} of $f$ and is said to {\bf
% extend}\index{extend|bi} $f$. In any diagrams, the presence of
% a dotted arrow or an arrow carrying a ? indicates a pious hope, in no way
% begging the question of its existence. Note that we shall usually
% omit $\circ$ from composite maps.

% \noindent
% {\bf The Lifting Problem}\index{lifting problem} \
% Given a pair of maps $E \stackrel{p}{\rightarrow}B$ and $X \stackrel{f}
% {\rightarrow} B $,
% does there exist a map $f^{\circ} : X \to E$, with
% $pf^{\circ} = f  $?


% That {\em all\/} existence problems about maps are essentially of one
% type or
% the other from these two is seen as follows. Evidently, all existence problems
% are representable by triangular diagrams\index{triangular diagrams} and it
% is easily seen that there are only these six possibilities:
% \begin{center}\begin{picture}(300,70)  %augch2 75
% \put(5,60){\vector(1,0){30}}
% \put(55,60){\vector(1,0){30}}
% \put(135,60){\vector(-1,0){30}}
% \put(185,60){\vector(-1,0){30}}
% \put(235,60){\vector(-1,0){30}}
% \put(285,60){\vector(-1,0){30}}
% \put(0,55){\vector(0,-1){30}}
% \put(50,55){\vector(0,-1){30}}
% \put(100,25){\vector(0,1){30}}
% \put(150,25){\vector(0,1){30}}
% \put(200,55){\vector(0,-1){30}}
% \put(250,55){\vector(0,-1){30}}
% \put(28,33){\small ?}
% \put(78,33){\small ?}
% \put(128,33){\small ?}
% \put(178,33){\small ?}
% \put(228,33){\small ?}
% \put(278,33){\small ?}
% \put(10,3){\bf 1}
% \put(60,3){\bf 2}
% \put(110,3){\bf 3}
% \put(160,3){\bf 4}
% \put(210,3){\bf 5}
% \put(260,3){\bf 6}
% \put(35,55){\vector(-1,-1){30}}
% \put(155,25){\vector(1,1){30}}
% \put(135,55){\vector(-1,-1){30}}
% \put(55,25){\vector(1,1){30}}
% \put(235,55){\vector(-1,-1){30}}
% \put(255,25){\vector(1,1){30}}
% \end{picture}\end{center}



% \begin{figure}
% \begin{picture}(300,220)(0,0)
% \put(-20,-20){\resizebox{20 cm}{!}{\includegraphics{3dpdf}}}
% \put(260,-10){\resizebox{15 cm}{!}{\includegraphics{contpdf}}}
% \put(220,80){$\beta$}
% \put(400,-10){$N$}
% \put(260,170){$\beta$}
% \put(90,15){$N$}
% \end{picture}
% \caption{{\em The log-gamma family of densities with central mean
% $<N> \, = \frac{1}{2}$ as a surface and as a contour plot. }}
% \label{pdf}
% \end{figure}

\newpage

%ch.tex


\chapter{Putting it all together}
\begin{center}
{\small\em The application as a whole}
\end{center}

\input{chapters/nine/putting-it-all-together}
\input{chapters/nine/end-to-end}
\input{chapters/nine/deployment}
\input{chapters/nine/application-to-framework}

% \section{Existence problems}
% We begin with some metamathematics.
% All problems about the existence of maps can be cast into one of the
% following two forms, which are in a sense mutually dual.

% \noindent
% {\bf The Extension Problem}\index{extension problem} \    %%% NB index entry tag
% Given an inclusion $ A \stackrel{i}{\hookrightarrow} X $, and a map
% $ A \stackrel{f}{\rightarrow} Y $,
% does there exist a map $f^{\dagger}:X\to Y$ such that
% $f^{\dagger}$ agrees with $f$ on $A$?

% Here the appropriate source category for maps should be clear from the
% context and, moreover, commutativity through a
% candidate $f^{\dagger}$ is precisely
% the restriction requirement; that is,
% $$f^{\dagger}   :  f^{\dagger}\circ i = f^{\dagger}|_A = f\,. $$
% If such an $f^{\dagger}$ exists\footnote{${}^{\dagger}$ suggests striving
% for perfection, crusading}, then it is called an {\bf
% extension}\index{extension!of a map|bi} of $f$ and is said to {\bf
% extend}\index{extend|bi} $f$. In any diagrams, the presence of
% a dotted arrow or an arrow carrying a ? indicates a pious hope, in no way
% begging the question of its existence. Note that we shall usually
% omit $\circ$ from composite maps.

% \noindent
% {\bf The Lifting Problem}\index{lifting problem} \
% Given a pair of maps $E \stackrel{p}{\rightarrow}B$ and $X \stackrel{f}
% {\rightarrow} B $,
% does there exist a map $f^{\circ} : X \to E$, with
% $pf^{\circ} = f  $?


% That {\em all\/} existence problems about maps are essentially of one
% type or
% the other from these two is seen as follows. Evidently, all existence problems
% are representable by triangular diagrams\index{triangular diagrams} and it
% is easily seen that there are only these six possibilities:
% \begin{center}\begin{picture}(300,70)  %augch2 75
% \put(5,60){\vector(1,0){30}}
% \put(55,60){\vector(1,0){30}}
% \put(135,60){\vector(-1,0){30}}
% \put(185,60){\vector(-1,0){30}}
% \put(235,60){\vector(-1,0){30}}
% \put(285,60){\vector(-1,0){30}}
% \put(0,55){\vector(0,-1){30}}
% \put(50,55){\vector(0,-1){30}}
% \put(100,25){\vector(0,1){30}}
% \put(150,25){\vector(0,1){30}}
% \put(200,55){\vector(0,-1){30}}
% \put(250,55){\vector(0,-1){30}}
% \put(28,33){\small ?}
% \put(78,33){\small ?}
% \put(128,33){\small ?}
% \put(178,33){\small ?}
% \put(228,33){\small ?}
% \put(278,33){\small ?}
% \put(10,3){\bf 1}
% \put(60,3){\bf 2}
% \put(110,3){\bf 3}
% \put(160,3){\bf 4}
% \put(210,3){\bf 5}
% \put(260,3){\bf 6}
% \put(35,55){\vector(-1,-1){30}}
% \put(155,25){\vector(1,1){30}}
% \put(135,55){\vector(-1,-1){30}}
% \put(55,25){\vector(1,1){30}}
% \put(235,55){\vector(-1,-1){30}}
% \put(255,25){\vector(1,1){30}}
% \end{picture}\end{center}



% \begin{figure}
% \begin{picture}(300,220)(0,0)
% \put(-20,-20){\resizebox{20 cm}{!}{\includegraphics{3dpdf}}}
% \put(260,-10){\resizebox{15 cm}{!}{\includegraphics{contpdf}}}
% \put(220,80){$\beta$}
% \put(400,-10){$N$}
% \put(260,170){$\beta$}
% \put(90,15){$N$}
% \end{picture}
% \caption{{\em The log-gamma family of densities with central mean
% $<N> \, = \frac{1}{2}$ as a surface and as a contour plot. }}
% \label{pdf}
% \end{figure}

\newpage

%ch.tex


\chapter{Putting it all together}
\begin{center}
{\small\em The application as a whole}
\end{center}

\input{chapters/nine/putting-it-all-together}
\input{chapters/nine/end-to-end}
\input{chapters/nine/deployment}
\input{chapters/nine/application-to-framework}

% \section{Existence problems}
% We begin with some metamathematics.
% All problems about the existence of maps can be cast into one of the
% following two forms, which are in a sense mutually dual.

% \noindent
% {\bf The Extension Problem}\index{extension problem} \    %%% NB index entry tag
% Given an inclusion $ A \stackrel{i}{\hookrightarrow} X $, and a map
% $ A \stackrel{f}{\rightarrow} Y $,
% does there exist a map $f^{\dagger}:X\to Y$ such that
% $f^{\dagger}$ agrees with $f$ on $A$?

% Here the appropriate source category for maps should be clear from the
% context and, moreover, commutativity through a
% candidate $f^{\dagger}$ is precisely
% the restriction requirement; that is,
% $$f^{\dagger}   :  f^{\dagger}\circ i = f^{\dagger}|_A = f\,. $$
% If such an $f^{\dagger}$ exists\footnote{${}^{\dagger}$ suggests striving
% for perfection, crusading}, then it is called an {\bf
% extension}\index{extension!of a map|bi} of $f$ and is said to {\bf
% extend}\index{extend|bi} $f$. In any diagrams, the presence of
% a dotted arrow or an arrow carrying a ? indicates a pious hope, in no way
% begging the question of its existence. Note that we shall usually
% omit $\circ$ from composite maps.

% \noindent
% {\bf The Lifting Problem}\index{lifting problem} \
% Given a pair of maps $E \stackrel{p}{\rightarrow}B$ and $X \stackrel{f}
% {\rightarrow} B $,
% does there exist a map $f^{\circ} : X \to E$, with
% $pf^{\circ} = f  $?


% That {\em all\/} existence problems about maps are essentially of one
% type or
% the other from these two is seen as follows. Evidently, all existence problems
% are representable by triangular diagrams\index{triangular diagrams} and it
% is easily seen that there are only these six possibilities:
% \begin{center}\begin{picture}(300,70)  %augch2 75
% \put(5,60){\vector(1,0){30}}
% \put(55,60){\vector(1,0){30}}
% \put(135,60){\vector(-1,0){30}}
% \put(185,60){\vector(-1,0){30}}
% \put(235,60){\vector(-1,0){30}}
% \put(285,60){\vector(-1,0){30}}
% \put(0,55){\vector(0,-1){30}}
% \put(50,55){\vector(0,-1){30}}
% \put(100,25){\vector(0,1){30}}
% \put(150,25){\vector(0,1){30}}
% \put(200,55){\vector(0,-1){30}}
% \put(250,55){\vector(0,-1){30}}
% \put(28,33){\small ?}
% \put(78,33){\small ?}
% \put(128,33){\small ?}
% \put(178,33){\small ?}
% \put(228,33){\small ?}
% \put(278,33){\small ?}
% \put(10,3){\bf 1}
% \put(60,3){\bf 2}
% \put(110,3){\bf 3}
% \put(160,3){\bf 4}
% \put(210,3){\bf 5}
% \put(260,3){\bf 6}
% \put(35,55){\vector(-1,-1){30}}
% \put(155,25){\vector(1,1){30}}
% \put(135,55){\vector(-1,-1){30}}
% \put(55,25){\vector(1,1){30}}
% \put(235,55){\vector(-1,-1){30}}
% \put(255,25){\vector(1,1){30}}
% \end{picture}\end{center}



% \begin{figure}
% \begin{picture}(300,220)(0,0)
% \put(-20,-20){\resizebox{20 cm}{!}{\includegraphics{3dpdf}}}
% \put(260,-10){\resizebox{15 cm}{!}{\includegraphics{contpdf}}}
% \put(220,80){$\beta$}
% \put(400,-10){$N$}
% \put(260,170){$\beta$}
% \put(90,15){$N$}
% \end{picture}
% \caption{{\em The log-gamma family of densities with central mean
% $<N> \, = \frac{1}{2}$ as a surface and as a contour plot. }}
% \label{pdf}
% \end{figure}

\newpage

%ch.tex


\chapter{Putting it all together}
\begin{center}
{\small\em The application as a whole}
\end{center}

\input{chapters/nine/putting-it-all-together}
\input{chapters/nine/end-to-end}
\input{chapters/nine/deployment}
\input{chapters/nine/application-to-framework}

% \section{Existence problems}
% We begin with some metamathematics.
% All problems about the existence of maps can be cast into one of the
% following two forms, which are in a sense mutually dual.

% \noindent
% {\bf The Extension Problem}\index{extension problem} \    %%% NB index entry tag
% Given an inclusion $ A \stackrel{i}{\hookrightarrow} X $, and a map
% $ A \stackrel{f}{\rightarrow} Y $,
% does there exist a map $f^{\dagger}:X\to Y$ such that
% $f^{\dagger}$ agrees with $f$ on $A$?

% Here the appropriate source category for maps should be clear from the
% context and, moreover, commutativity through a
% candidate $f^{\dagger}$ is precisely
% the restriction requirement; that is,
% $$f^{\dagger}   :  f^{\dagger}\circ i = f^{\dagger}|_A = f\,. $$
% If such an $f^{\dagger}$ exists\footnote{${}^{\dagger}$ suggests striving
% for perfection, crusading}, then it is called an {\bf
% extension}\index{extension!of a map|bi} of $f$ and is said to {\bf
% extend}\index{extend|bi} $f$. In any diagrams, the presence of
% a dotted arrow or an arrow carrying a ? indicates a pious hope, in no way
% begging the question of its existence. Note that we shall usually
% omit $\circ$ from composite maps.

% \noindent
% {\bf The Lifting Problem}\index{lifting problem} \
% Given a pair of maps $E \stackrel{p}{\rightarrow}B$ and $X \stackrel{f}
% {\rightarrow} B $,
% does there exist a map $f^{\circ} : X \to E$, with
% $pf^{\circ} = f  $?


% That {\em all\/} existence problems about maps are essentially of one
% type or
% the other from these two is seen as follows. Evidently, all existence problems
% are representable by triangular diagrams\index{triangular diagrams} and it
% is easily seen that there are only these six possibilities:
% \begin{center}\begin{picture}(300,70)  %augch2 75
% \put(5,60){\vector(1,0){30}}
% \put(55,60){\vector(1,0){30}}
% \put(135,60){\vector(-1,0){30}}
% \put(185,60){\vector(-1,0){30}}
% \put(235,60){\vector(-1,0){30}}
% \put(285,60){\vector(-1,0){30}}
% \put(0,55){\vector(0,-1){30}}
% \put(50,55){\vector(0,-1){30}}
% \put(100,25){\vector(0,1){30}}
% \put(150,25){\vector(0,1){30}}
% \put(200,55){\vector(0,-1){30}}
% \put(250,55){\vector(0,-1){30}}
% \put(28,33){\small ?}
% \put(78,33){\small ?}
% \put(128,33){\small ?}
% \put(178,33){\small ?}
% \put(228,33){\small ?}
% \put(278,33){\small ?}
% \put(10,3){\bf 1}
% \put(60,3){\bf 2}
% \put(110,3){\bf 3}
% \put(160,3){\bf 4}
% \put(210,3){\bf 5}
% \put(260,3){\bf 6}
% \put(35,55){\vector(-1,-1){30}}
% \put(155,25){\vector(1,1){30}}
% \put(135,55){\vector(-1,-1){30}}
% \put(55,25){\vector(1,1){30}}
% \put(235,55){\vector(-1,-1){30}}
% \put(255,25){\vector(1,1){30}}
% \end{picture}\end{center}



% \begin{figure}
% \begin{picture}(300,220)(0,0)
% \put(-20,-20){\resizebox{20 cm}{!}{\includegraphics{3dpdf}}}
% \put(260,-10){\resizebox{15 cm}{!}{\includegraphics{contpdf}}}
% \put(220,80){$\beta$}
% \put(400,-10){$N$}
% \put(260,170){$\beta$}
% \put(90,15){$N$}
% \end{picture}
% \caption{{\em The log-gamma family of densities with central mean
% $<N> \, = \frac{1}{2}$ as a surface and as a contour plot. }}
% \label{pdf}
% \end{figure}

\newpage

%ch.tex


\chapter{Putting it all together}
\begin{center}
{\small\em The application as a whole}
\end{center}

\input{chapters/nine/putting-it-all-together}
\input{chapters/nine/end-to-end}
\input{chapters/nine/deployment}
\input{chapters/nine/application-to-framework}

% \section{Existence problems}
% We begin with some metamathematics.
% All problems about the existence of maps can be cast into one of the
% following two forms, which are in a sense mutually dual.

% \noindent
% {\bf The Extension Problem}\index{extension problem} \    %%% NB index entry tag
% Given an inclusion $ A \stackrel{i}{\hookrightarrow} X $, and a map
% $ A \stackrel{f}{\rightarrow} Y $,
% does there exist a map $f^{\dagger}:X\to Y$ such that
% $f^{\dagger}$ agrees with $f$ on $A$?

% Here the appropriate source category for maps should be clear from the
% context and, moreover, commutativity through a
% candidate $f^{\dagger}$ is precisely
% the restriction requirement; that is,
% $$f^{\dagger}   :  f^{\dagger}\circ i = f^{\dagger}|_A = f\,. $$
% If such an $f^{\dagger}$ exists\footnote{${}^{\dagger}$ suggests striving
% for perfection, crusading}, then it is called an {\bf
% extension}\index{extension!of a map|bi} of $f$ and is said to {\bf
% extend}\index{extend|bi} $f$. In any diagrams, the presence of
% a dotted arrow or an arrow carrying a ? indicates a pious hope, in no way
% begging the question of its existence. Note that we shall usually
% omit $\circ$ from composite maps.

% \noindent
% {\bf The Lifting Problem}\index{lifting problem} \
% Given a pair of maps $E \stackrel{p}{\rightarrow}B$ and $X \stackrel{f}
% {\rightarrow} B $,
% does there exist a map $f^{\circ} : X \to E$, with
% $pf^{\circ} = f  $?


% That {\em all\/} existence problems about maps are essentially of one
% type or
% the other from these two is seen as follows. Evidently, all existence problems
% are representable by triangular diagrams\index{triangular diagrams} and it
% is easily seen that there are only these six possibilities:
% \begin{center}\begin{picture}(300,70)  %augch2 75
% \put(5,60){\vector(1,0){30}}
% \put(55,60){\vector(1,0){30}}
% \put(135,60){\vector(-1,0){30}}
% \put(185,60){\vector(-1,0){30}}
% \put(235,60){\vector(-1,0){30}}
% \put(285,60){\vector(-1,0){30}}
% \put(0,55){\vector(0,-1){30}}
% \put(50,55){\vector(0,-1){30}}
% \put(100,25){\vector(0,1){30}}
% \put(150,25){\vector(0,1){30}}
% \put(200,55){\vector(0,-1){30}}
% \put(250,55){\vector(0,-1){30}}
% \put(28,33){\small ?}
% \put(78,33){\small ?}
% \put(128,33){\small ?}
% \put(178,33){\small ?}
% \put(228,33){\small ?}
% \put(278,33){\small ?}
% \put(10,3){\bf 1}
% \put(60,3){\bf 2}
% \put(110,3){\bf 3}
% \put(160,3){\bf 4}
% \put(210,3){\bf 5}
% \put(260,3){\bf 6}
% \put(35,55){\vector(-1,-1){30}}
% \put(155,25){\vector(1,1){30}}
% \put(135,55){\vector(-1,-1){30}}
% \put(55,25){\vector(1,1){30}}
% \put(235,55){\vector(-1,-1){30}}
% \put(255,25){\vector(1,1){30}}
% \end{picture}\end{center}



% \begin{figure}
% \begin{picture}(300,220)(0,0)
% \put(-20,-20){\resizebox{20 cm}{!}{\includegraphics{3dpdf}}}
% \put(260,-10){\resizebox{15 cm}{!}{\includegraphics{contpdf}}}
% \put(220,80){$\beta$}
% \put(400,-10){$N$}
% \put(260,170){$\beta$}
% \put(90,15){$N$}
% \end{picture}
% \caption{{\em The log-gamma family of densities with central mean
% $<N> \, = \frac{1}{2}$ as a surface and as a contour plot. }}
% \label{pdf}
% \end{figure}

\newpage

%ch.tex


\chapter{Putting it all together}
\begin{center}
{\small\em The application as a whole}
\end{center}

\input{chapters/nine/putting-it-all-together}
\input{chapters/nine/end-to-end}
\input{chapters/nine/deployment}
\input{chapters/nine/application-to-framework}

% \section{Existence problems}
% We begin with some metamathematics.
% All problems about the existence of maps can be cast into one of the
% following two forms, which are in a sense mutually dual.

% \noindent
% {\bf The Extension Problem}\index{extension problem} \    %%% NB index entry tag
% Given an inclusion $ A \stackrel{i}{\hookrightarrow} X $, and a map
% $ A \stackrel{f}{\rightarrow} Y $,
% does there exist a map $f^{\dagger}:X\to Y$ such that
% $f^{\dagger}$ agrees with $f$ on $A$?

% Here the appropriate source category for maps should be clear from the
% context and, moreover, commutativity through a
% candidate $f^{\dagger}$ is precisely
% the restriction requirement; that is,
% $$f^{\dagger}   :  f^{\dagger}\circ i = f^{\dagger}|_A = f\,. $$
% If such an $f^{\dagger}$ exists\footnote{${}^{\dagger}$ suggests striving
% for perfection, crusading}, then it is called an {\bf
% extension}\index{extension!of a map|bi} of $f$ and is said to {\bf
% extend}\index{extend|bi} $f$. In any diagrams, the presence of
% a dotted arrow or an arrow carrying a ? indicates a pious hope, in no way
% begging the question of its existence. Note that we shall usually
% omit $\circ$ from composite maps.

% \noindent
% {\bf The Lifting Problem}\index{lifting problem} \
% Given a pair of maps $E \stackrel{p}{\rightarrow}B$ and $X \stackrel{f}
% {\rightarrow} B $,
% does there exist a map $f^{\circ} : X \to E$, with
% $pf^{\circ} = f  $?


% That {\em all\/} existence problems about maps are essentially of one
% type or
% the other from these two is seen as follows. Evidently, all existence problems
% are representable by triangular diagrams\index{triangular diagrams} and it
% is easily seen that there are only these six possibilities:
% \begin{center}\begin{picture}(300,70)  %augch2 75
% \put(5,60){\vector(1,0){30}}
% \put(55,60){\vector(1,0){30}}
% \put(135,60){\vector(-1,0){30}}
% \put(185,60){\vector(-1,0){30}}
% \put(235,60){\vector(-1,0){30}}
% \put(285,60){\vector(-1,0){30}}
% \put(0,55){\vector(0,-1){30}}
% \put(50,55){\vector(0,-1){30}}
% \put(100,25){\vector(0,1){30}}
% \put(150,25){\vector(0,1){30}}
% \put(200,55){\vector(0,-1){30}}
% \put(250,55){\vector(0,-1){30}}
% \put(28,33){\small ?}
% \put(78,33){\small ?}
% \put(128,33){\small ?}
% \put(178,33){\small ?}
% \put(228,33){\small ?}
% \put(278,33){\small ?}
% \put(10,3){\bf 1}
% \put(60,3){\bf 2}
% \put(110,3){\bf 3}
% \put(160,3){\bf 4}
% \put(210,3){\bf 5}
% \put(260,3){\bf 6}
% \put(35,55){\vector(-1,-1){30}}
% \put(155,25){\vector(1,1){30}}
% \put(135,55){\vector(-1,-1){30}}
% \put(55,25){\vector(1,1){30}}
% \put(235,55){\vector(-1,-1){30}}
% \put(255,25){\vector(1,1){30}}
% \end{picture}\end{center}



% \begin{figure}
% \begin{picture}(300,220)(0,0)
% \put(-20,-20){\resizebox{20 cm}{!}{\includegraphics{3dpdf}}}
% \put(260,-10){\resizebox{15 cm}{!}{\includegraphics{contpdf}}}
% \put(220,80){$\beta$}
% \put(400,-10){$N$}
% \put(260,170){$\beta$}
% \put(90,15){$N$}
% \end{picture}
% \caption{{\em The log-gamma family of densities with central mean
% $<N> \, = \frac{1}{2}$ as a surface and as a contour plot. }}
% \label{pdf}
% \end{figure}

\newpage

%ch.tex


\chapter{Putting it all together}
\begin{center}
{\small\em The application as a whole}
\end{center}

\input{chapters/nine/putting-it-all-together}
\input{chapters/nine/end-to-end}
\input{chapters/nine/deployment}
\input{chapters/nine/application-to-framework}

% \section{Existence problems}
% We begin with some metamathematics.
% All problems about the existence of maps can be cast into one of the
% following two forms, which are in a sense mutually dual.

% \noindent
% {\bf The Extension Problem}\index{extension problem} \    %%% NB index entry tag
% Given an inclusion $ A \stackrel{i}{\hookrightarrow} X $, and a map
% $ A \stackrel{f}{\rightarrow} Y $,
% does there exist a map $f^{\dagger}:X\to Y$ such that
% $f^{\dagger}$ agrees with $f$ on $A$?

% Here the appropriate source category for maps should be clear from the
% context and, moreover, commutativity through a
% candidate $f^{\dagger}$ is precisely
% the restriction requirement; that is,
% $$f^{\dagger}   :  f^{\dagger}\circ i = f^{\dagger}|_A = f\,. $$
% If such an $f^{\dagger}$ exists\footnote{${}^{\dagger}$ suggests striving
% for perfection, crusading}, then it is called an {\bf
% extension}\index{extension!of a map|bi} of $f$ and is said to {\bf
% extend}\index{extend|bi} $f$. In any diagrams, the presence of
% a dotted arrow or an arrow carrying a ? indicates a pious hope, in no way
% begging the question of its existence. Note that we shall usually
% omit $\circ$ from composite maps.

% \noindent
% {\bf The Lifting Problem}\index{lifting problem} \
% Given a pair of maps $E \stackrel{p}{\rightarrow}B$ and $X \stackrel{f}
% {\rightarrow} B $,
% does there exist a map $f^{\circ} : X \to E$, with
% $pf^{\circ} = f  $?


% That {\em all\/} existence problems about maps are essentially of one
% type or
% the other from these two is seen as follows. Evidently, all existence problems
% are representable by triangular diagrams\index{triangular diagrams} and it
% is easily seen that there are only these six possibilities:
% \begin{center}\begin{picture}(300,70)  %augch2 75
% \put(5,60){\vector(1,0){30}}
% \put(55,60){\vector(1,0){30}}
% \put(135,60){\vector(-1,0){30}}
% \put(185,60){\vector(-1,0){30}}
% \put(235,60){\vector(-1,0){30}}
% \put(285,60){\vector(-1,0){30}}
% \put(0,55){\vector(0,-1){30}}
% \put(50,55){\vector(0,-1){30}}
% \put(100,25){\vector(0,1){30}}
% \put(150,25){\vector(0,1){30}}
% \put(200,55){\vector(0,-1){30}}
% \put(250,55){\vector(0,-1){30}}
% \put(28,33){\small ?}
% \put(78,33){\small ?}
% \put(128,33){\small ?}
% \put(178,33){\small ?}
% \put(228,33){\small ?}
% \put(278,33){\small ?}
% \put(10,3){\bf 1}
% \put(60,3){\bf 2}
% \put(110,3){\bf 3}
% \put(160,3){\bf 4}
% \put(210,3){\bf 5}
% \put(260,3){\bf 6}
% \put(35,55){\vector(-1,-1){30}}
% \put(155,25){\vector(1,1){30}}
% \put(135,55){\vector(-1,-1){30}}
% \put(55,25){\vector(1,1){30}}
% \put(235,55){\vector(-1,-1){30}}
% \put(255,25){\vector(1,1){30}}
% \end{picture}\end{center}



% \begin{figure}
% \begin{picture}(300,220)(0,0)
% \put(-20,-20){\resizebox{20 cm}{!}{\includegraphics{3dpdf}}}
% \put(260,-10){\resizebox{15 cm}{!}{\includegraphics{contpdf}}}
% \put(220,80){$\beta$}
% \put(400,-10){$N$}
% \put(260,170){$\beta$}
% \put(90,15){$N$}
% \end{picture}
% \caption{{\em The log-gamma family of densities with central mean
% $<N> \, = \frac{1}{2}$ as a surface and as a contour plot. }}
% \label{pdf}
% \end{figure}

\newpage

%ch.tex


\chapter{Putting it all together}
\begin{center}
{\small\em The application as a whole}
\end{center}

\input{chapters/nine/putting-it-all-together}
\input{chapters/nine/end-to-end}
\input{chapters/nine/deployment}
\input{chapters/nine/application-to-framework}

% \section{Existence problems}
% We begin with some metamathematics.
% All problems about the existence of maps can be cast into one of the
% following two forms, which are in a sense mutually dual.

% \noindent
% {\bf The Extension Problem}\index{extension problem} \    %%% NB index entry tag
% Given an inclusion $ A \stackrel{i}{\hookrightarrow} X $, and a map
% $ A \stackrel{f}{\rightarrow} Y $,
% does there exist a map $f^{\dagger}:X\to Y$ such that
% $f^{\dagger}$ agrees with $f$ on $A$?

% Here the appropriate source category for maps should be clear from the
% context and, moreover, commutativity through a
% candidate $f^{\dagger}$ is precisely
% the restriction requirement; that is,
% $$f^{\dagger}   :  f^{\dagger}\circ i = f^{\dagger}|_A = f\,. $$
% If such an $f^{\dagger}$ exists\footnote{${}^{\dagger}$ suggests striving
% for perfection, crusading}, then it is called an {\bf
% extension}\index{extension!of a map|bi} of $f$ and is said to {\bf
% extend}\index{extend|bi} $f$. In any diagrams, the presence of
% a dotted arrow or an arrow carrying a ? indicates a pious hope, in no way
% begging the question of its existence. Note that we shall usually
% omit $\circ$ from composite maps.

% \noindent
% {\bf The Lifting Problem}\index{lifting problem} \
% Given a pair of maps $E \stackrel{p}{\rightarrow}B$ and $X \stackrel{f}
% {\rightarrow} B $,
% does there exist a map $f^{\circ} : X \to E$, with
% $pf^{\circ} = f  $?


% That {\em all\/} existence problems about maps are essentially of one
% type or
% the other from these two is seen as follows. Evidently, all existence problems
% are representable by triangular diagrams\index{triangular diagrams} and it
% is easily seen that there are only these six possibilities:
% \begin{center}\begin{picture}(300,70)  %augch2 75
% \put(5,60){\vector(1,0){30}}
% \put(55,60){\vector(1,0){30}}
% \put(135,60){\vector(-1,0){30}}
% \put(185,60){\vector(-1,0){30}}
% \put(235,60){\vector(-1,0){30}}
% \put(285,60){\vector(-1,0){30}}
% \put(0,55){\vector(0,-1){30}}
% \put(50,55){\vector(0,-1){30}}
% \put(100,25){\vector(0,1){30}}
% \put(150,25){\vector(0,1){30}}
% \put(200,55){\vector(0,-1){30}}
% \put(250,55){\vector(0,-1){30}}
% \put(28,33){\small ?}
% \put(78,33){\small ?}
% \put(128,33){\small ?}
% \put(178,33){\small ?}
% \put(228,33){\small ?}
% \put(278,33){\small ?}
% \put(10,3){\bf 1}
% \put(60,3){\bf 2}
% \put(110,3){\bf 3}
% \put(160,3){\bf 4}
% \put(210,3){\bf 5}
% \put(260,3){\bf 6}
% \put(35,55){\vector(-1,-1){30}}
% \put(155,25){\vector(1,1){30}}
% \put(135,55){\vector(-1,-1){30}}
% \put(55,25){\vector(1,1){30}}
% \put(235,55){\vector(-1,-1){30}}
% \put(255,25){\vector(1,1){30}}
% \end{picture}\end{center}



% \begin{figure}
% \begin{picture}(300,220)(0,0)
% \put(-20,-20){\resizebox{20 cm}{!}{\includegraphics{3dpdf}}}
% \put(260,-10){\resizebox{15 cm}{!}{\includegraphics{contpdf}}}
% \put(220,80){$\beta$}
% \put(400,-10){$N$}
% \put(260,170){$\beta$}
% \put(90,15){$N$}
% \end{picture}
% \caption{{\em The log-gamma family of densities with central mean
% $<N> \, = \frac{1}{2}$ as a surface and as a contour plot. }}
% \label{pdf}
% \end{figure}

\newpage

%ch.tex


\chapter{Putting it all together}
\begin{center}
{\small\em The application as a whole}
\end{center}

\input{chapters/nine/putting-it-all-together}
\input{chapters/nine/end-to-end}
\input{chapters/nine/deployment}
\input{chapters/nine/application-to-framework}

% \section{Existence problems}
% We begin with some metamathematics.
% All problems about the existence of maps can be cast into one of the
% following two forms, which are in a sense mutually dual.

% \noindent
% {\bf The Extension Problem}\index{extension problem} \    %%% NB index entry tag
% Given an inclusion $ A \stackrel{i}{\hookrightarrow} X $, and a map
% $ A \stackrel{f}{\rightarrow} Y $,
% does there exist a map $f^{\dagger}:X\to Y$ such that
% $f^{\dagger}$ agrees with $f$ on $A$?

% Here the appropriate source category for maps should be clear from the
% context and, moreover, commutativity through a
% candidate $f^{\dagger}$ is precisely
% the restriction requirement; that is,
% $$f^{\dagger}   :  f^{\dagger}\circ i = f^{\dagger}|_A = f\,. $$
% If such an $f^{\dagger}$ exists\footnote{${}^{\dagger}$ suggests striving
% for perfection, crusading}, then it is called an {\bf
% extension}\index{extension!of a map|bi} of $f$ and is said to {\bf
% extend}\index{extend|bi} $f$. In any diagrams, the presence of
% a dotted arrow or an arrow carrying a ? indicates a pious hope, in no way
% begging the question of its existence. Note that we shall usually
% omit $\circ$ from composite maps.

% \noindent
% {\bf The Lifting Problem}\index{lifting problem} \
% Given a pair of maps $E \stackrel{p}{\rightarrow}B$ and $X \stackrel{f}
% {\rightarrow} B $,
% does there exist a map $f^{\circ} : X \to E$, with
% $pf^{\circ} = f  $?


% That {\em all\/} existence problems about maps are essentially of one
% type or
% the other from these two is seen as follows. Evidently, all existence problems
% are representable by triangular diagrams\index{triangular diagrams} and it
% is easily seen that there are only these six possibilities:
% \begin{center}\begin{picture}(300,70)  %augch2 75
% \put(5,60){\vector(1,0){30}}
% \put(55,60){\vector(1,0){30}}
% \put(135,60){\vector(-1,0){30}}
% \put(185,60){\vector(-1,0){30}}
% \put(235,60){\vector(-1,0){30}}
% \put(285,60){\vector(-1,0){30}}
% \put(0,55){\vector(0,-1){30}}
% \put(50,55){\vector(0,-1){30}}
% \put(100,25){\vector(0,1){30}}
% \put(150,25){\vector(0,1){30}}
% \put(200,55){\vector(0,-1){30}}
% \put(250,55){\vector(0,-1){30}}
% \put(28,33){\small ?}
% \put(78,33){\small ?}
% \put(128,33){\small ?}
% \put(178,33){\small ?}
% \put(228,33){\small ?}
% \put(278,33){\small ?}
% \put(10,3){\bf 1}
% \put(60,3){\bf 2}
% \put(110,3){\bf 3}
% \put(160,3){\bf 4}
% \put(210,3){\bf 5}
% \put(260,3){\bf 6}
% \put(35,55){\vector(-1,-1){30}}
% \put(155,25){\vector(1,1){30}}
% \put(135,55){\vector(-1,-1){30}}
% \put(55,25){\vector(1,1){30}}
% \put(235,55){\vector(-1,-1){30}}
% \put(255,25){\vector(1,1){30}}
% \end{picture}\end{center}



% \begin{figure}
% \begin{picture}(300,220)(0,0)
% \put(-20,-20){\resizebox{20 cm}{!}{\includegraphics{3dpdf}}}
% \put(260,-10){\resizebox{15 cm}{!}{\includegraphics{contpdf}}}
% \put(220,80){$\beta$}
% \put(400,-10){$N$}
% \put(260,170){$\beta$}
% \put(90,15){$N$}
% \end{picture}
% \caption{{\em The log-gamma family of densities with central mean
% $<N> \, = \frac{1}{2}$ as a surface and as a contour plot. }}
% \label{pdf}
% \end{figure}

\newpage

%ch.tex


\chapter{Putting it all together}
\begin{center}
{\small\em The application as a whole}
\end{center}

\input{chapters/nine/putting-it-all-together}
\input{chapters/nine/end-to-end}
\input{chapters/nine/deployment}
\input{chapters/nine/application-to-framework}

% \section{Existence problems}
% We begin with some metamathematics.
% All problems about the existence of maps can be cast into one of the
% following two forms, which are in a sense mutually dual.

% \noindent
% {\bf The Extension Problem}\index{extension problem} \    %%% NB index entry tag
% Given an inclusion $ A \stackrel{i}{\hookrightarrow} X $, and a map
% $ A \stackrel{f}{\rightarrow} Y $,
% does there exist a map $f^{\dagger}:X\to Y$ such that
% $f^{\dagger}$ agrees with $f$ on $A$?

% Here the appropriate source category for maps should be clear from the
% context and, moreover, commutativity through a
% candidate $f^{\dagger}$ is precisely
% the restriction requirement; that is,
% $$f^{\dagger}   :  f^{\dagger}\circ i = f^{\dagger}|_A = f\,. $$
% If such an $f^{\dagger}$ exists\footnote{${}^{\dagger}$ suggests striving
% for perfection, crusading}, then it is called an {\bf
% extension}\index{extension!of a map|bi} of $f$ and is said to {\bf
% extend}\index{extend|bi} $f$. In any diagrams, the presence of
% a dotted arrow or an arrow carrying a ? indicates a pious hope, in no way
% begging the question of its existence. Note that we shall usually
% omit $\circ$ from composite maps.

% \noindent
% {\bf The Lifting Problem}\index{lifting problem} \
% Given a pair of maps $E \stackrel{p}{\rightarrow}B$ and $X \stackrel{f}
% {\rightarrow} B $,
% does there exist a map $f^{\circ} : X \to E$, with
% $pf^{\circ} = f  $?


% That {\em all\/} existence problems about maps are essentially of one
% type or
% the other from these two is seen as follows. Evidently, all existence problems
% are representable by triangular diagrams\index{triangular diagrams} and it
% is easily seen that there are only these six possibilities:
% \begin{center}\begin{picture}(300,70)  %augch2 75
% \put(5,60){\vector(1,0){30}}
% \put(55,60){\vector(1,0){30}}
% \put(135,60){\vector(-1,0){30}}
% \put(185,60){\vector(-1,0){30}}
% \put(235,60){\vector(-1,0){30}}
% \put(285,60){\vector(-1,0){30}}
% \put(0,55){\vector(0,-1){30}}
% \put(50,55){\vector(0,-1){30}}
% \put(100,25){\vector(0,1){30}}
% \put(150,25){\vector(0,1){30}}
% \put(200,55){\vector(0,-1){30}}
% \put(250,55){\vector(0,-1){30}}
% \put(28,33){\small ?}
% \put(78,33){\small ?}
% \put(128,33){\small ?}
% \put(178,33){\small ?}
% \put(228,33){\small ?}
% \put(278,33){\small ?}
% \put(10,3){\bf 1}
% \put(60,3){\bf 2}
% \put(110,3){\bf 3}
% \put(160,3){\bf 4}
% \put(210,3){\bf 5}
% \put(260,3){\bf 6}
% \put(35,55){\vector(-1,-1){30}}
% \put(155,25){\vector(1,1){30}}
% \put(135,55){\vector(-1,-1){30}}
% \put(55,25){\vector(1,1){30}}
% \put(235,55){\vector(-1,-1){30}}
% \put(255,25){\vector(1,1){30}}
% \end{picture}\end{center}



% \begin{figure}
% \begin{picture}(300,220)(0,0)
% \put(-20,-20){\resizebox{20 cm}{!}{\includegraphics{3dpdf}}}
% \put(260,-10){\resizebox{15 cm}{!}{\includegraphics{contpdf}}}
% \put(220,80){$\beta$}
% \put(400,-10){$N$}
% \put(260,170){$\beta$}
% \put(90,15){$N$}
% \end{picture}
% \caption{{\em The log-gamma family of densities with central mean
% $<N> \, = \frac{1}{2}$ as a surface and as a contour plot. }}
% \label{pdf}
% \end{figure}

\newpage



\documentclass[12pt,leqno]{book}
\usepackage{amsmath,amssymb,amsfonts} % Typical maths resource packages
\usepackage{graphicx}                 % Packages to allow inclusion of graphics
\usepackage{color}                    % For creating coloured text and background
\usepackage{hyperref}                 % For creating hyperlinks in cross references
\usepackage{makeidx}                  % For indexing
\usepackage{listings}                 % For code listing
\usepackage{mathpartir}               % For grammars, rules, etc
\usepackage{bcprules}                 % For other kinds of rules
\usepackage{diagrams}                 % For commutative diagrams

\lstloadlanguages{Scala,Java,Haskell,XML,bash,HTML,SQL}

\parindent 1cm
\parskip 0.2cm
\topmargin 0.2cm
\oddsidemargin 1cm
\evensidemargin 0.5cm
\textwidth 15cm
\textheight 21cm

\include{local/local}

\makeindex


\title{Pro Scala: Monadic Design Patterns for the Web}

\author{L.G. Meredith  \\
{\small\em \copyright \  Draft date \today }}

 \date{ }
\begin{document}
\lstset{language=Haskell}
\maketitle
 \addcontentsline{toc}{chapter}{Contents}
\pagenumbering{roman}
\tableofcontents
\listoffigures
\listoftables
\chapter*{Preface}\normalsize
  \addcontentsline{toc}{chapter}{Preface}
\pagestyle{plain}
% The book root file {\tt bookex.tex} gives a basic example of how to
% use \LaTeX \ for preparation of a book. Note that all
% \LaTeX \ commands begin with a
% backslash.

% Each
% Chapter, Appendix and the Index is made as a {\tt *.tex} file and is
% called in by the {\tt include} command---thus {\tt ch1.tex} is
% the name here of the file containing Chapter~1. The inclusion of any
% particular file can be suppressed by prefixing the line by a
% percent sign.


%  Do not put an {\tt end{document}} command at the end of chapter files;
% just one such command is needed at the end of the book.

% Note the tag used to make an index entry. You may need to consult Lamport's
% book~\cite{lamport} for details of the procedure to make the index input
% file; \LaTeX \ will create a pre-index by listing all the tagged
% items in the file {\tt bookex.idx} then you edit this into
% a {\tt theindex} environment, as {\tt index.tex}.

The book you hold in your hands, Dear Reader, is not at all what you expected...



\pagestyle{headings}
\pagenumbering{arabic}

\include{chapters/one/ch}
\include{chapters/two/ch}
\include{chapters/three/ch}
\include{chapters/four/ch}
\include{chapters/five/ch}
\include{chapters/six/ch}
\include{chapters/seven/ch}
\include{chapters/eight/ch}
\include{chapters/nine/ch}
\include{chapters/ten/ch}

\include{bibliography/monadic}

%\include{index/index}
  \addcontentsline{toc}{chapter}{Index}
\end{document}


%\index{monad!categorical}
\index{monad!haskell}
  \addcontentsline{toc}{chapter}{Index}
\end{document}


%\index{monad!categorical}
\index{monad!haskell}
  \addcontentsline{toc}{chapter}{Index}
\end{document}


%\index{monad!categorical}
\index{monad!haskell}
  \addcontentsline{toc}{chapter}{Index}
\end{document}
